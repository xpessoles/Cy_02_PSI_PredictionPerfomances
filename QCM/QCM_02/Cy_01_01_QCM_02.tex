\documentclass[10pt,fleqn]{article} % Default font size and left-justified equations
\usepackage[%
    pdftitle={Modélisation systèmes multiphysiques : Modélisation linéaire et non linéaire},
    pdfauthor={Xavier Pessoles}]{hyperref}
    
\input{style/new_style}
\input{style/macros_SII}
\usepackage{multicol}
\usepackage{siunitx}
\fichetrue
%\fichefalse

\proftrue
%\proffalse

\tdtrue
%\tdfalse

\courstrue
\coursfalse

\def\discipline{Sciences \\Industrielles de \\ l'Ingénieur}
\def\xxtete{Sciences Industrielles de l'Ingénieur}

\def\classe{PSI$\star$ -- MP}
\def\xxnumpartie{Cycle 02}
\def\xxpartie{Modéliser les systèmes asservis dans le but de prévoir leur comportement}


\def\xxnumchapitre{Chapitre 3 \vspace{.2cm}}
\def\xxchapitre{\hspace{.12cm} Précision des systèmes}


\def\xxtitreexo{QCM}
\def\xxsourceexo{\hspace{.2cm} \footnotesize{X. Pessoles}}


\def\xxposongletx{2}
\def\xxposonglettext{1.45}
\def\xxposonglety{20}
%\def\xxonglet{Part. 1 -- Ch. 3}
\def\xxonglet{\textsf{Cycle 02}}

\def\xxactivite{QCM 02}
\def\xxauteur{\textsl{X. Pessoles}}

\def\xxcompetences{%
\textsl{%
\textbf{Savoirs et compétences :}\\
%Les sources sont associées par un \emph{hacheur série}. La détermination des grandeurs électriques associées à ce montage permet de conclure vis à vis du cahier des charges.
%\noindent \textbf{Résoudre :} à partir des modèles retenus :
%\begin{itemize}[label=\ding{112},font=\color{ocre}] 
%\item choisir une méthode de résolution analytique, graphique, numérique;
%\item mettre en \oe{}uvre une méthode de résolution.
%\end{itemize}
%\begin{itemize}[label=\ding{112},font=\color{ocre}] 
%\item \textit{Rés -- C1.1 :} Loi entrée sortie géométrique et cinématique -- Fermeture géométrique.
%\end{itemize}
%
%\noindent \textit{Mod2 -- C4.1 :} Représentation par schéma bloc.
}}

\def\xxfigures{
%\includegraphics[width=.9\linewidth]{images/c-evolution}
}%figues de la page de garde


\def\xxpied{%
Cycle 02 -- Modéliser les SLCI afin de prévoir leur comportement\\
Chapitre 3 -- \xxactivite%
}

\setcounter{secnumdepth}{5}
%---------------------------------------------------------------------------

\usepackage{pgfplots}
\begin{document}
\def\pathfig{images}
%\chapterimage{png/Fond_Cin}
\input{style/new_pagegarde}
\vspace{4.5cm}
\pagestyle{fancy}
\thispagestyle{plain}

\def\columnseprulecolor{\color{ocre}}
\setlength{\columnseprule}{0.4pt} 

\def\pathfig{images}

\begin{multicols}{2}
\begin{enumerate}
\item Soit un système dont la FTBO est d'ordre 2 et de gain K.
\begin{enumerate}
\item L'écart statique est nul. (Échelon unitaire)
\item L'écart statique vaut 1/(K+1). (Échelon un.)
\item L'écart statique vaut 1/K. (Échelon unitaire)
\item L'écart statique est infini. (Échelon unitaire)
\item L'écart statique est K-1.
\item On ne peut pas conclure. % OUI
\end{enumerate}
\item Soit un système dont la FTBF est un système du second ordre 2 (K/(1+2zp/om+p²/om²))et de gain K.
\begin{enumerate}
\item L'écart statique est nul. (Échelon unitaire)
\item L'écart statique vaut 1/(K+1). (Échelon un.)
\item L'écart statique vaut 1/K. (Échelon unitaire)
\item L'écart statique est infini. (Échelon unitaire)
\item L'écart statique est K-1.% OUI
\item On ne peut pas conclure. 
\end{enumerate}
\item Soit un système dont la FTBO est de classe 0.
\begin{enumerate}
\item L'écart statique est nul. (Échelon unitaire)
\item L'écart statique vaut 1/(K+1). (Échelon un.) % OUI
\item L'écart statique vaut 1/K. (Échelon unitaire) 
\item L'écart statique est infini. (Échelon unitaire)
\item L'écart statique est K-1.
\item On ne peut pas conclure.  
\end{enumerate}
\item  Soit un système dont la FTBO est de classe 1.
\begin{enumerate}
\item L'erreur de traînage est nulle. (rampe de pente 1)
\item L'erreur de traînage vaut 1/(K+1). (rampe de pente 1)
\item L'erreur de traînage vaut 1/K. (rampe de pente 1) %OUI
\item L'erreur de traînage est infinie. (rampe de pente 1)
\item L'erreur de traînage est K-1. (rampe de pente 1)
\item On ne peut pas conclure.
\end{enumerate}
\item Soit un système dont la FTBO est de classe 0.
\begin{enumerate}
\item L'erreur de trainage est nulle. (rampe de pente 1)
\item L'erreur de traînage vaut 1/(K+1). (rampe de pente 1)
\item L'erreur de traînage vaut 1/K. (rampe de pente~1)
\item L'erreur de traînage est infinie. (rampe de pente~1) %OUI
\item L'erreur de traînage est K-1. (rampe de pente~1)
\item On ne peut pas conclure.
\end{enumerate}
\item Soit un système dont la FTBO est de classe 2.
\begin{enumerate}
\item L'erreur de trainage est nulle. (rampe de pente 1)%OUI
\item L'erreur de traînage vaut 1/(K+1). (rampe de pente 1)
\item L'erreur de traînage vaut 1/K. (rampe de pente 1)
\item L'erreur de traînage est infinie. (rampe de pente 1) 
\item L'erreur de traînage est K-1. (rampe de pente 1)
\item On ne peut pas conclure.
\end{enumerate}
\item Soit un système dont la FTBO est de classe 2.
\begin{enumerate}
\item L'écart statique est nul. (Échelon unitaire)% OUI
\item L'écart statique vaut 1/(K+1). (Échelon unitaire) 
\item L'écart statique vaut 1/K. (Échelon unitaire) 
\item L'écart statique est infini. (Échelon unitaire)
\item L'écart statique est K-1.
\item On ne peut pas conclure.  
\end{enumerate}
\item Soit un système dont la FTBO est de classe 1.
\begin{enumerate}
\item L'écart statique est nul. (Échelon unitaire)% OUI
\item L'écart statique vaut 1/(K+1). (Échelon unitaire) 
\item L'écart statique vaut 1/K. (Échelon unitaire) 
\item L'écart statique est infini. (Échelon unitaire)
\item L'écart statique est K-1.
\item On ne peut pas conclure.  
\end{enumerate}
\item Soit F(p) la fonction de transfert en boucle ouverte n'admettant aucune intégration. L'erreur en régime statique est :
\begin{enumerate}
\item constante non nulle
\item nulle
\item infinie
\end{enumerate}
\item Soit F(p) la fonction de transfert en boucle ouverte admettant une seule intégration. L'erreur en régime statique est :
\begin{enumerate}
\item constante non nulle
\item nulle % OUI
\item infinie
\end{enumerate}
La précision des systèmes asservis vis-à-vis d'une perturbation dépend t-elle du point d'application de cette perturbation?
\begin{enumerate}
\item OUI
\item NON
\end{enumerate}
\end{enumerate}

\end{multicols}







\end{document}