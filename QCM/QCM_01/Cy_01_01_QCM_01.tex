\documentclass[10pt,fleqn]{article} % Default font size and left-justified equations
\usepackage[%
    pdftitle={Modélisation systèmes multiphysiques : Modélisation linéaire et non linéaire},
    pdfauthor={Xavier Pessoles}]{hyperref}
    
\input{style/new_style}
\input{style/macros_SII}
\usepackage{multicol}
\usepackage{siunitx}
\fichetrue
%\fichefalse

\proftrue
\proffalse

\tdtrue
%\tdfalse

\courstrue
\coursfalse

\def\discipline{Sciences \\Industrielles de \\ l'Ingénieur}
\def\xxtete{Sciences Industrielles de l'Ingénieur}

\def\classe{PSI$\star$ -- MP}
\def\xxnumpartie{Cycle 02}
\def\xxpartie{Modéliser les systèmes asservis dans le but de prévoir leur comportement}


\def\xxnumchapitre{Chapitre 2 \vspace{.2cm}}
\def\xxchapitre{\hspace{.12cm} Rapidité des systèmes}


\def\xxtitreexo{QCM}
\def\xxsourceexo{\hspace{.2cm} \footnotesize{Éd Vuibert.}}


\def\xxposongletx{2}
\def\xxposonglettext{1.45}
\def\xxposonglety{20}
%\def\xxonglet{Part. 1 -- Ch. 3}
\def\xxonglet{\textsf{Cycle 02}}

\def\xxactivite{QCM 01}
\def\xxauteur{\textsl{Éd Vuibert.}}

\def\xxcompetences{%
\textsl{%
\textbf{Savoirs et compétences :}\\
%Les sources sont associées par un \emph{hacheur série}. La détermination des grandeurs électriques associées à ce montage permet de conclure vis à vis du cahier des charges.
%\noindent \textbf{Résoudre :} à partir des modèles retenus :
%\begin{itemize}[label=\ding{112},font=\color{ocre}] 
%\item choisir une méthode de résolution analytique, graphique, numérique;
%\item mettre en \oe{}uvre une méthode de résolution.
%\end{itemize}
%\begin{itemize}[label=\ding{112},font=\color{ocre}] 
%\item \textit{Rés -- C1.1 :} Loi entrée sortie géométrique et cinématique -- Fermeture géométrique.
%\end{itemize}
%
%\noindent \textit{Mod2 -- C4.1 :} Représentation par schéma bloc.
}}

\def\xxfigures{
%\includegraphics[width=.9\linewidth]{images/c-evolution}
}%figues de la page de garde


\def\xxpied{%
Cycle 02 -- Modéliser les SLCI dans le but de prévoir leur comportement\\
Chapitre 2 -- \xxactivite%
}

\setcounter{secnumdepth}{5}
%---------------------------------------------------------------------------

\usepackage{pgfplots}
\begin{document}
\def\pathfig{images}
%\chapterimage{png/Fond_Cin}
\input{style/new_pagegarde}
\vspace{4.5cm}
\pagestyle{fancy}
\thispagestyle{plain}

\def\columnseprulecolor{\color{ocre}}
\setlength{\columnseprule}{0.4pt} 

\def\pathfig{images}

\begin{multicols}{2}
%\end{multicols}
\ifprof
\else
Vrai ou faux : 
\begin{enumerate}
\item Le temps de réponse à 5\% d’une FTBF d’ordre 3 présentant les pôles $-2$ et $-12\pm 5j$ est d’environ \SI{1,5}{s}, car $-2$ est le pôle dominant. % VRAI
\item Une FTBO du deuxième ordre fortement résonante conduira à une marge de gain très faible, voire à un système instable en boucle fermée. % FAUX
\item Une FTBF d’ordre 3 présentant les pôles $-8$ et$-3 \pm 10$ (rad/s) sera très oscillante mais rapide. %Vrai
\item  Un système présentant une marge de gain élevée de \SI{15}{dB} mais une marge de phase faible de 10\degres sera peu stable. %Vrai
\item Lorsque la FTBO est elle-même instable, la FTBF sera nécessairement instable, car les marges ne sont pas définies. %Faux
\item Une FTBF d’ordre 3 présentant les pôles 4 et $-8\pm 5 $(rad/s) est stable. %Faux
\end{enumerate}

\begin{enumerate}
\setcounter{enumi}{6}
\item On donne le diagramme de Bode de la FTBF d'un système : 
\begin{center}
\includegraphics[width=\linewidth]{images/img_01}
\end{center}
\begin{enumerate}
\item Le système est stable parce que la marge de phase est positive et la marge de gain est infinie.
\item Le système est instable parce que la marge de phase est négative et la marge de gain est infinie.
\item La marge de gain est positive.
\item La marge de gain est négative.
\item La marge de phase est positive.
\item La marge de phase est négative.
\item Le système est stable.
\item Le système est instable.
\item On ne peut pas conclure. % OUI
\end{enumerate}

\item On donne le diagramme de Bode de la FTBO d'un système : 
\begin{center}
\includegraphics[width=\linewidth]{images/img_02}
\end{center}
\begin{enumerate}
\item Le système est stable parce que la marge de phase est positive et la marge de gain est infinie.
\item Le système est instable parce que la marge de phase est négative.
\item La marge de gain est positive. % OUI
\item La marge de gain est négative. % OUI
\item La marge de phase est positive.
\item La marge de phase est négative.
\item Le système est stable.
\item Le système est instable. % OUI
\item On ne peut pas conclure. % 
\end{enumerate}
\item On donne le diagramme de Bode de la FTBF d'un système : 
\begin{center}
\includegraphics[width=\linewidth]{images/img_03}
\end{center}
\begin{enumerate}
\item Le système est stable parce que la marge de phase est positive et la marge de gain est infinie.
\item Le système est instable parce que la marge de phase est négative et la marge de gain est infinie.
\item La marge de gain est positive.
\item La marge de gain est négative.
\item La marge de phase est positive.
\item La marge de phase est négative.
\item Le système est stable.
\item Le système est instable.
\item On ne peut pas conclure. % OUI
\end{enumerate}

\item On donne le diagramme de Bode de la FTBO d'un système : 
\begin{center}
\includegraphics[width=\linewidth]{images/img_04}
\end{center}
\begin{enumerate}
\item Le système est stable parce que la marge de phase est positive et la marge de gain est infinie.
\item Le système est instable parce que la marge de phase est négative et la marge de gain est infinie.
\item La marge de gain est positive. % OUI
\item La marge de gain est négative. % 
\item La marge de phase est positive. % OUI
\item La marge de phase est négative.
\item Le système est stable. % OUI
\item Le système est instable.
\item On ne peut pas conclure. % NON
\end{enumerate}
\end{enumerate}


\end{multicols}

\fi

\ifprof
\begin{corrige}
\begin{enumerate}
\item Vrai. Les pôles complexes sont beaucoup plus rapides et donc auront convergé bien avant
1,5 s. Le pôle -2 est dominant et la constante de temps associée vaut \SI{0,5}{s}. Le temps de réponse
d’un premier ordre avec cette constante de temps vaut $3\times0,5 = \SI{1,5}{s}$.
\item Faux. Un deuxième ordre en boucle fermée restera un deuxième ordre à coefficients positifs,
donc stable. Par ailleurs, la phase d’un deuxième ordre n’atteignant jamais 180\degres, la marge de gain
n’est pas définie (ou infinie) et ne peut pas conduire à l’instabilité. Néanmoins, pour une FTBO
du deuxième ordre résonant dont le gain dépasse \SI{0}{dB}, la forme du diagramme de Bode conduit
à croiser l’axe \SI{0}{dB} après la résonance. Or, pour un système fortement résonant, la phase chute
brutalement à $\omega_0$ en direction de -180°. Il y a donc toutes les chances pour que la marge de phase
soit très petite, donc le système en boucle fermée peu stable, donc convergent mais très oscillant.
\item Vrai, pour les oscillations. Effectivement, les pôles dominants sont les pôles complexes (partie
réelle la plus faible) et la partie imaginaire est bien plus importante en module que la partie réelle,
ce qui indique une fréquence d’oscillation élevée au regard du temps de convergence. Par contre,
il est difficile de dire si le système sera rapide : le temps de réponse est court ou long au regard
d’un autre temps caractéristique, mais n’est pas court en soi !
\item Vrai. Si une des deux marges est faible, la courbe passe « près » du point critique -1 et la FTBF
présente alors une résonance. Une bonne stabilité est assurée si les deux marges sont suffisantes.
\item Faux. Une FTBO instable peut très bien être stabilisée par la boucle d’asservissement, ce qui
est le cas des deux exemples du cours. Néanmoins, la caractérisation de la
stabilité à partir du critère du revers n’est plus possible, car ce critère pose comme hypothèse de
départ que la FTBO est stable.... 
\item Faux. Il suffit d’un pôle à partie réelle positive pour que la réponse temporelle à une sollicitation
d’entrée présente un terme divergent, et donc diverge.
\end{enumerate}
\end{corrige}
\else
\fi







\end{document}