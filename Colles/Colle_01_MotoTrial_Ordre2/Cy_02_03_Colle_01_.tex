\documentclass[10pt,fleqn]{article} % Default font size and left-justified equations
\usepackage[%
    pdftitle={Modélisation SLCI : Rapidité des systèmes},
    pdfauthor={Xavier Pessoles}]{hyperref}
    
\input{style/new_style}
\input{style/macros_SII}
\usepackage{multicol}
\usepackage{siunitx}
%\usepackage{picins}
\fichetrue
%\fichefalse

\proftrue
\proffalse

\tdtrue
%\tdfalse

\courstrue
\coursfalse

\def\discipline{Sciences \\Industrielles de \\ l'Ingénieur}
\def\xxtete{Sciences Industrielles de l'Ingénieur}

\def\classe{PSI$\star$ -- MP}
\def\xxnumpartie{Cycle 02}
\def\xxpartie{Modéliser les systèmes asservis dans le but de prévoir leur comportement}


\def\xxnumchapitre{Chapitre 2 \vspace{.2cm}}
\def\xxchapitre{\hspace{.12cm} Rapidité des systèmes}


\def\xxtitreexo{Moto de trial électrique}
\def\xxsourceexo{\hspace{.2cm} \footnotesize{E3A MP}}


\def\xxposongletx{2}
\def\xxposonglettext{1.45}
\def\xxposonglety{20}
%\def\xxonglet{Part. 1 -- Ch. 3}
\def\xxonglet{Cycle 02}

\def\xxactivite{TD 01}
\def\xxauteur{\textsl{Éditions Vuibert}}

\def\xxcompetences{%
\textsl{%
\textbf{Savoirs et compétences :}\\
%Les sources sont associées par un \emph{hacheur série}. La détermination des grandeurs électriques associées à ce montage permet de conclure vis à vis du cahier des charges.
%\noindent \textbf{Résoudre :} à partir des modèles retenus :
%\begin{itemize}[label=\ding{112},font=\color{ocre}] 
%\item choisir une méthode de résolution analytique, graphique, numérique;
%\item mettre en \oe{}uvre une méthode de résolution.
%\end{itemize}
%\begin{itemize}[label=\ding{112},font=\color{ocre}] 
%\item \textit{Rés -- C1.1 :} Loi entrée sortie géométrique et cinématique -- Fermeture géométrique.
%\end{itemize}
%
%\noindent \textit{Mod2 -- C4.1 :} Représentation par schéma bloc.
}}

\def\xxfigures{
\includegraphics[width=.5\linewidth]{images/fig_01}
}%figues de la page de garde


\def\xxpied{%
Cycle 01 -- Modéliser le comportement des systèmes multiphysiques\\
Chapitre 2 -- \xxactivite%
}

\setcounter{secnumdepth}{5}
%---------------------------------------------------------------------------

\usepackage{pgfplots}
\begin{document}

%\chapterimage{png/Fond_Cin}
\input{style/new_pagegarde}
\vspace{5cm}
\pagestyle{fancy}
\thispagestyle{plain}

\def\columnseprulecolor{\color{ocre}}
\setlength{\columnseprule}{0.4pt} 

\def\pathfig{images}

\begin{multicols}{2}


\subsection*{Mise en situation}
\subsubsection*{Présentation}
%\noindent \begin{minipage}[c]{.7\linewidth}
La motorisation électrique fait désormais partie intégrante du paysage des deux-roues motorisés.
À l'image de l'industrie automobile, la propulsion électrique est le nouveau cheval de bataille de nombreux
constructeurs de 2 roues, voire l'unique alternative aux soucis de pollution qu’elle soit chimique ou sonore. La
culture d’entreprise d’Electric-Motion est essentiellement tournée vers l’électrique.

À l’heure actuelle l’offre moto électrique est réduite et les gammes sont plus que restreintes.
Electric-Motion étend la gamme des possibilités offertes aux amateurs de 2 roues en proposant un modèle trial
aux adeptes de « green motorcycle ».
%\end{minipage} \hfill
%\begin{minipage}[c]{.28\linewidth}
%\begin{center}
%	\includegraphics[width=\linewidth]{images/fig_01}
% 	%\textit{Diagramme de Bode (gain uniquement) du système corrigé.}
%\end{center}
%\end{minipage}

\subsubsection*{Extrait des exigences fonctionnelles}
%
\begin{center}
	\includegraphics[width=\linewidth]{images/fig_02_bis}
\end{center}
%
%
%\subsection*{Vérification de la vitesse de la moto}
%
%\begin{obj}
%Bien que la moto soit un modèle trial, une vitesse minimale doit être atteinte pour un
%minimum de confort lors des courts déplacements. Nous allons vérifier cette performance.
%\end{obj}
%
%La moto est équipée de 2 transmissions en série, la transmission primaire par courroie crantée, et la
%transmission secondaire par chaîne. Les caractéristiques des transmissions sont données dans le tableau suivant :
%
%\begin{center}
%\begin{tabular}{|l|l|l|}
%\hline
%Courroie primaire & $Z_{P1}=\SI{20}{dents}$ & $Z_{P2}=\SI{44}{dents}$ \\
%\hline
%Chaîne secondaire & $Z_{S1}=\SI{9}{dents}$ & $Z_{s2}=\SI{57}{dents}$ \\
%\hline
%\end{tabular}
%\end{center}
%
%\begin{center}
%	\includegraphics[width=\linewidth]{images/fig_03}
%\end{center}
%
%Le rayon de la roue arrière est $R_1=\SI{345}{mm}$ (pneu compris).  
%La fréquence de rotation maxi du moteur vaut : $N_{\text{max}}=\SI{4200}{tr/min}$. 
%
%\subparagraph{}
%\textit{Déterminer la vitesse de la moto en m/s puis en km/h, conclure vis-à-vis des exigences
%fonctionnelles.}
%\ifprof
%\begin{corrige}
%\end{corrige}
%\else
%\fi




\subsection*{Vérification des performances de la suspension}

\begin{obj}
Justifier le choix par le constructeur de l’amortisseur arrière et de son réglage.
\end{obj}

La figure ci-dessous représente %une vue CAO de la suspension et
 le modèle retenu pour l'amortisseur. 

\begin{center}
%\begin{tabular}{c}
%	\includegraphics[width=6cm]{images/fig_04} \\
       \includegraphics[width=6cm]{images/fig_05}

%\end{tabular}
\end{center}

On suppose que l’origine de $x(t)$ correspond à la situation où la moto roule avec un pilote dessus, en l’absence de défaut de la route. $y(t)$ caractérise le profil de la route.
L’équation de la résultante dynamique appliquée à la masse donne donc :
$$M\ddot{x}(t)= - k\left(x(t)-y(t)\right) -\mu \left(\dot{x}(t)-\dot{y}(t) \right).$$

On notera $f(t)$ en temporel et $F(p)$ sa transformée dans le domaine de Laplace.



\subparagraph{}
\textit{Exprimer la fonction de transfert de l’amortisseur $H(p)=\dfrac{X(p)}{Y(p)}$
sous forme canonique.}
\ifprof
\begin{corrige}
\end{corrige}
\else
\fi

Pour la suite, on néglige le terme en p du numérateur.


\subparagraph{}
\textit{Déterminer la pulsation propre du système $\omega_0$, le gain $K_a$ et le facteur d’amortissement $z$ en
fonction de $M$, $k$ et $\mu$.}
\ifprof
\begin{corrige}
\end{corrige}
\else
\fi

Lors de l’étude de l’équilibre de la moto, la masse de l’ensemble est répartie équitablement entre la roue avant et la roue arrière. On donne la masse $M=\SI{70}{kg}$ de l’ensemble comprenant la moitié de la masse moto+pilote, et
la raideur du ressort $k=\SI{70000}{N/m}$.

\subparagraph{}
\textit{Choisir le coefficient d’amortissement $\mu$ pour avoir un temps de réponse à 5\% minimal.}
\ifprof
\begin{corrige}
\end{corrige}
\else
\fi

%\subsection*{Réponse de la suspension à un obstacle}
%La moto rencontre un obstacle modélisé par un échelon $y_0$ de $\SI{50}{mm}$ du profil de la route. On cherche à en tracer la réponse.
%
%Données numériques : on donne $M=\SI{70}{kg}$, $k=\SI{70000}{N/m}$ et $\mu=\SI{3000}{N.s/m}$.
%
%$$
%M\ddot{x}(t)=-k\left(x(t)-y(t) \right)-\mu\left(\dot{x}(t)-\dot{y}(t) \right)+ky_0.
%$$
%
%\subparagraph{}
%\textit{Déterminer $a$, $b$ et $c$ numériquement tel que : $\ddot{x}(t)=a\dot{x}(t)+bx(t)+c$.}
%\ifprof
%\begin{corrige}
%\end{corrige}
%\else
%\fi


\subsection*{Influence de la pente sur la vitesse maximale de la moto}
%\begin{obj}

%Modélisation du système de motorisation et influence de la pente sur l’accélération et la%
vitesse maximale.
%\end{obj}

%
%
%\begin{center}
%\begin{tabular}{|p{3.5cm}|p{3.5cm}|p{3.5cm}|p{3.5cm}|}
%\hline 
%Composant & Données && \\
%\hline
%Moto +Utilisateur & $Mmu= \SI{170}{kg}$ && \\ \hline
%Roue AV  & Inertie : $J_{av} = \SI{0.8}{kg.m^2}$ &  Diamètre : \SI{690}{mm} &\\ \hline
%Roue AR & Inertie : $J_{ar} = \SI{0.3}{kg.m^2}$ &  Diamètre : \SI{690}{mm} &\\ \hline
%Rotor & Inertie : $J_r =\SI{0.006}{kg.m^2}$ & Diamètre : \SI{155}{mm} & \\ \hline
%Arbre intermédiaire & Inertie :$J_{int} = \SI{0.0017}{kg.m^2}$ & &  \\ \hline
%Courroie primaire : & $Z_{p1} = 20$ &  $Z_{p2} =44$ &  Rapport de transmission :
%$K1 = 0.45$  \\ \hline
%Chaine secondaire : & $Z_{s1} = 9$ & $Z_{s2} =57 $ & Rapport de transmission :
%$K2 = 0.16$ \\ \hline
%Moteur synchrone,
%assimilable à un MCC : & 
%$R_m= 0,15 \Omega $
%
%$L_m=\SI{60}{mH}$
%
%Constantes du moteur :
%
%$K_e=\SI{0,12}{V/rad/s}$
%
%$K_m=\SI{0,12}{N.m/A}$
%
%$K=K_m=K_e$ 
%&
%Intensité maxi :
%100A en régime
%permanent
%220A en pic
%&U=\SI{48}{V}\\ \hline
%%& $U=\SI{48}{V}$ \\ \hline
%Batterie :&  
%I max=220A 
%
% $U=\SI{48}{V}$ 
%  & & 
% \\ \hline
%\end{tabular}
%\end{center}

Le pilote demande une consigne en tension $U_c$ au moteur à l’aide de la poignée d’accélérateur (comprise entre 0 et 48V). Le moteur va donc créer sur la poulie $P_1$ un couple $C_m$. On souhaite connaître la vitesse à laquelle peut aller la moto en fonction de la pente. On aura donc, une consigne $u_c(t)$ en volt et une réponse $\omega_m (t)$ en rad/s.

On se placera dans différents cas :
\begin{itemize}
\item sur le plat : $Cr=\SI{0}{Nm}$;
\item sur une faible pente (20\%) : $Cr = \SI{110}{Nm}$;
\item sur une forte pente (40\%) : $Cr = \SI{210}{Nm}$.
\end{itemize}

Hypothèses :
\begin{itemize}
\item on suppose que les roues de la moto restent en contact avec le sol, sans glisser;
\item on appelle l’ensemble $ \Sigma$=\{Moto + Pilote + roueAV + roueAR + Arbre intermédiaire+ Rotor\}.
\end{itemize}
Le schéma-blocs suivant modélise la commande en vitesse du moteur :
\begin{center}
	\includegraphics[width=.9\linewidth]{images/fig_06}
\end{center}


En appliquant le théorème de l'énergie cinétique, on obtient : $J_{eq}\dot{\omega}_m(t)=C_r(t) K_1 K_2 + C_m(t)$.

\subparagraph{}
\textit{En déduire les fonctions de transfert $H4(p)$ et $H1(p)$ littéralement.}
\ifprof
\begin{corrige}
\end{corrige}
\else
\fi

Pour la suite du sujet, on prendra $J_{eq} = {0,1}{kg.m^2}$. On assimile ce moteur brushless à un moteur à courant continu. Les équations du comportement du moteur
sont donc :
$u_m(t) =  R_m i_m(t) + L_m \dfrac{di_m(t)}{dt} + e(t)$, $e (t)= K_e \omega_m (t)$, 
$C_m(t) = K_mi(t)$.



\subparagraph{}
\textit{En déduire les fonctions de transfert : $H_2(p)$, $H_3(p)$ et $H_5(p)$.}
\ifprof
\begin{corrige}
\end{corrige}
\else
\fi


\subparagraph{}
\textit{Montrer que l’on peut écrire  $\Omega_m(p)$ sous la forme :
$\Omega_m(p)=H_u(p)Uc(p)-H_{Cr}(p)Cr(p)$.
%$\Omega_m(p)=H_u(p)Uc(p)-H_{Cr}(p)C_r(p)$.
Pour cela, expliciter $H_{cr}(p)$ et $H_u(p)$ en fonction des $H_1(p)$, $H_2(p)$, …$H_5(p)$.
}
\ifprof
\begin{corrige}
\end{corrige}
\else
\fi


\subparagraph{}
\textit{Dans le cas où $Cr(p) =0$, déterminer la fonction de transfert du moteur en Boucle Fermée $H_U(p)=\dfrac{\Omega_m(p)}{U_c(p)}$ sous la forme $H_u(p)=\dfrac{K_v}{1+\dfrac{2z}{\omega_0}p+\dfrac{p^2}{\omega_0^2}}$.}

\ifprof
\begin{corrige}
\end{corrige}
\else
\fi


\subparagraph{}
\textit{Calculer les valeur littérales de $Kv$, $z$ et $\omega_0$, puis faire l’application numérique.}
\ifprof
\begin{corrige}
\end{corrige}
\else
\fi

Pour la suite on prendra des valeurs suivantes : $z=0,8$ et $\omega_0 = \SI{1,55}{rad/s}$.



\subparagraph{}
\textit{De quel ordre est ce système ? Calculer le temps de réponse à 5\% et la valeur du premier
dépassement de ce système à l’aide des abaques fournis ci-dessous. Conclure vis-à-vis des
exigences fonctionnelles.}
\ifprof
\begin{corrige}
\end{corrige}
\else
\fi

\vspace{.25cm}

\begin{center}
\includegraphics[width=\linewidth]{images/fig_07}
\end{center}

\begin{center}
\includegraphics[width=\linewidth]{images/fig_08}
\end{center}


\vspace{.25cm}

On rappelle la relation de la pseudo-période $T$ d’une réponse indicielle d’un système de second ordre : $T=\dfrac{2\pi}{\omega_0 \sqrt{1-z^2}}$.


\subparagraph{}
\textit{Compléter sur le document réponse la courbe de réponse en vitesse pour un $Cr=\SI{0}{Nm}$ en indiquant sur la courbe les tangentes, le temps du 1\ier dépassement, le premier dépassement, et le $Tr_{5\%}$.}

\ifprof
\begin{corrige}
\end{corrige}
\else
\fi


On a sur la courbe du document réponse, la réponse à un échelon de commande (accélérateur à fond) de la moto qui roule à plat et pour 2 pentes différentes (20\% et 40\%).

\subparagraph{}
\textit{Le temps d’accélération est-il changé ? Sur quelle valeur influence ce changement de pente ? Conclure vis-à-vis des exigences fonctionnelles.}
\ifprof
\begin{corrige}
\end{corrige}
\else
\fi

\end{multicols}
%
%
%\newpage
%
%\begin{center}
%%\includegraphics[width=\linewidth]{images/cor_01}
%%\textit{}
%\end{center}
%
%
%\begin{center}
%%\includegraphics[width=\linewidth]{images/cor_02}
%%\textit{}
%\end{center}


\begin{center}
	\includegraphics[width=16cm]{images/dr_01}
\end{center}

\newpage


\begin{center}
	\includegraphics[width=.9\linewidth]{images/cor_01}
\end{center}

\begin{center}
	\includegraphics[width=.9\linewidth]{images/cor_02}
\end{center}

\begin{center}
	\includegraphics[width=.9\linewidth]{images/cor_03}
\end{center}

\begin{center}
	\includegraphics[width=.9\linewidth]{images/cor_04}
\end{center}

\end{document}

\subparagraph{}\textit{}


\begin{center}
\includegraphics[width=\linewidth]{images/fig_06}
%\textit{}
\end{center}
\begin{center}
\includegraphics[width=\linewidth]{images/img_04}
%\textit{}
\end{center}

