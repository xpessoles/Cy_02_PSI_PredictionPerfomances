\documentclass[10pt,fleqn]{article} % Default font size and left-justified equations
\usepackage[%
    pdftitle={Modélisation SLCI : Théorie},
    pdfauthor={Xavier Pessoles}]{hyperref}

\input{style/new_style}
\input{style/macros_SII}

\fichetrue
\fichefalse

\proftrue
%\proffalse

%\tdtrue
\tdfalse

\courstrue
%\coursfalse



% -------------------------------------
% Déclaration des titres
% -------------------------------------

\def\discipline{Sciences \\Industrielles de \\ l'Ingénieur}
\def\xxtete{Sciences Industrielles de l'Ingénieur}

\def\classe{PSI$\star$ -- MP}
\def\xxnumpartie{Cycle 02}
\def\xxpartie{Modéliser les systèmes asservis dans le but de prévoir leur comportement}

\def\xxnumchapitre{Cours \vspace{.2cm}}
\def\xxchapitre{\hspace{.12cm} Théorie des SLCI}

\def\xxposongletx{2}
\def\xxposonglettext{1.45}
\def\xxposonglety{19}%16

\def\xxonglet{Cycle 02}

\def\xxactivite{Cours}
\def\xxauteur{\textsl{Xavier Pessoles}}

\def\xxcompetences{%
\textsl{%
\textbf{Savoirs et compétences :}\\
\begin{itemize}[label=\ding{112},font=\color{ocre}] 
\item \textit{Mod3.C2 : } pôles dominants et réduction de l’ordre du modèle : principe, justification
\item \textit{Res2.C4 : } stabilité des SLCI : définition entrée bornée -- sortie bornée (EB -- SB)	
\item \textit{Res2.C5 : } stabilité des SLCI : équation caractéristique	
\item \textit{Res2.C6 : } stabilité des SLCI : position des pôles dans le plan complexe
\item \textit{Res2.C7 : } stabilité des SLCI : marges de stabilité (de gain et de phase)
\end{itemize}
}}

	
		
	
		


\def\xxfigures{
\includegraphics[width=3cm]{images/SoloWheel_Orbit}%images/prot_01
\\
\textit{SoloWheel Orbit.}
}%figues de la page de garde

\def\xxpied{%
Cycle 02 -- Modéliser les SLCI dans le but de prévoir leur comportement\\
Chapitre 1 -- \xxactivite%
}

\setcounter{secnumdepth}{5}
%---------------------------------------------------------------------------


\begin{document}
\chapterimage{png/Fond_SLCI}
\input{style/new_pagegarde}
\setlength{\columnseprule}{.1pt}

\vspace{2cm}
\pagestyle{fancy}
\thispagestyle{plain}
\section{Stabilité des sysètmes}
\subsection{Notion de stabilité}
\subsubsection{Représentation graphique \cite{1}}
Un état d'équilibre d'un système est asymptotiquement stable lorsque le système, écarté de sa position d'équilibre par une
cause extérieure, finit par retrouver ce même état d'équilibre après disparition de la
cause.
Illustrons cette définition de façon très intuitive à travers l'exemple suivant : une boule
soumise à l'accélération de la pesanteur se déplaçant (avec un peu de dissipation
énergétique) sur une surface donnée.

\begin{center}
\includegraphics[width=.9\linewidth]{images/fig_stabilite}
\end{center}
\subsubsection{Premières définitions}
\begin{defi}[Définition intuitive]

Un système est asymptotiquement stable si et seulement si : 
\begin{itemize}
\item abandonné à lui-même à partir de conditions initiales quelconques il revient à son état d'équilibre;
\item son régime transitoire finit par disparaître;
\item sa sortie finit par ressembler à l'entrée;
\item sa réponse tend vers zéro au cours du temps.
\end{itemize}

\end{defi}

\begin{rem}
La stabilité d'un système \textbf{est indépendante} de la nature de l'entrée. Ainsi, l'étude de la stabilité peut se faire à partir d'une réponse impulsionnelle (entrée Dirac), indicielle (entrée échelon d'amplitude 1), d'une réponse harmonique (entrée sinusoïdale)...

Pour simplifier les calculs, une première approche pourra être d'utiliser la réponse impulsionnelle. 
\end{rem}
\begin{defi}
En conséquence, on peut considérer qu'un système est asymptotiquement stable si et seulement si sa réponse impulsionnelle tend vers zéro au cours du temps.
\end{defi}

\subsubsection{Étude des pôles de la fonction de transfert}
Dans le cas général la fonction de transfert d'un système peut se mettre sous la forme :
$$
H(p)=\dfrac{b_mp^m + b_{m-1}p^{m-1}+...+b_1p+b_0}{a_np^n + a_{n-1}p^{n-1}+...+a_1p+a_0} \quad \text{avec } n\geq m.
$$

Lors du calcul de la réponse temporelle en utilisant la transformée de Laplace inverse (quelle que soit l'entrée), la nature du régime transitoire ne dépend que des pôles $p_i$de la fonction de transfert (zéros du dénominateur).

En factorisant le numérateur et le dénominateur de $H(p)$ on peut alors retrouver une fonction de la forme  :
$$
H(p)=\dfrac{\left(p+ z_m\right)\cdot \left(p+ z_{m-1}\right)...}{\left(p+ p_n\right)\cdot \left(p+ p_{n-1}\right)...} \quad \text{avec } p_i,z_i\in \mathbb{C}.
$$

En passant dans le domaine temporel : 
\begin{itemize}
\item les pôles réels (de type $p=-a$) induisent des modes\footnote{mode : fonction temporelle associée à un pôle} du type $e^{-at}$;
\item les pôles complexes conjugués (de type $p=-a\pm j\omega$) induisent des modes du type 
$e^{-at} \sin \omega t$.
\end{itemize}

\textbf{On peut ainsi constater que si les pôles sont à partie réelle strictement négative, l'exponentielle décroissante permet de stabiliser la réponse temporelle.}

Ainsi, on peut observer la réponse temporelle des systèmes en fonction du positionnement des pôles dans le plan complexe. 

\begin{center}
\includegraphics[width=14cm]{images/poles_simple_double}

\textit{Représentation d'un système à pôle simple et à pôles conjugués dans le plan complexe -- Réponse indicielle}
\end{center}

\subsubsection{Position des pôles dans le plan complexe}
Par extension on peut observer dans le plan complexe les pôles de fonctions de transfert et leur indicielle associée.

\begin{center}
\includegraphics[width=\linewidth]{images/plan_complexe_fm}

\textit{Allure de la réponse à l’impulsion de Dirac selon la position des pôles de la FTBF d’un système \cite{2}.}
\end{center}

\begin{defi}[À retenir]
Un système est asymptotiquement stable si et seulement si tous les pôles de sa fonction de transfert (\textbf{en boucle fermée}) sont à partie réelle strictement négative. 

\end{defi}

\begin{rem}On peut montrer que :
\begin{itemize}
\item \textbf{pour les systèmes d'ordre 1 et 2 :} le système est stable si tous les coefficients du dénominateur sont non nuls et de même signe;
\item \textbf{pour les systèmes d'ordre 3 :} de la forme $a_0+a_1p+a_2p^2+a_3p^3$ les coefficients doivent être strictement de même signe et $a_2 a_1 > a_3 a_0$.
\end{itemize}
\end{rem}

\subsubsection{Pôles dominants \cite{1}}
Lors de l’étude d’un système, on se contente en général de ne prendre en compte que les pôles les plus influents. Ces pôles sont appelés les pôles dominants. Pour un système asymptotiquement stable, ce sont ceux qui sont le plus proche de l’axe des imaginaires, puisque ce sont eux qui induisent des modes qui disparaissent dans le temps le plus lentement.

\subsubsection{Caractéristiques dans le lieu de pôles}
Il est possible de représenter les performances des systèmes asservis en utilisant le lieu des pôles dans le plan complexe \cite{1}. 

\subsection{Marges de stabilité}
\subsubsection{Lorsque la BO commence à pointer le bout de son nez...}


Soit le schéma-blocs suivant : 

\begin{center}
\includestandalone{images/Schema_1_entree_F_R}
\end{center}

La fonction de transfert en boucle ouverte est donnée par $H_{BO}=\dfrac{R(p)}{\varepsilon(p)}=F(p)G(p)$. 

La fonction de transfert en boucle fermée est donnée par : $H_{BF}=\dfrac{S(p)}{E(p)}=\dfrac{F(p)}{1+F(p)G(p)}=\dfrac{F(p)}{1+H_{BO}(p)}$. 

\begin{defi}[Équation caractéristique]
Soit $H(p)=\dfrac{N(p)}{D(p)}$ une fonction de transfert. On appelle $D(p)=0$ l'équation caractéristique de la fonction de transfert. Ainsi les racines de $D(p)$ correspondent aux pôles de $H(p)$.
\end{defi}

\textbf{Pour un système bouclé, l'équation caractéristique sera $1+H_{BO}(p)=0$.}

\subsubsection{Critère algébrique de stabilité : le critère de Routh}
Pour un système d'ordre supérieur à 3 il devient délicat d'obtenir analytiquement (ou numériquement) les racines du polynôme et ainsi conclure sur la stabilité à partir du signe des parties réelles. 

Il existe un critère algébrique permettant de vérifier la stabilité d'un système : il s'agit de critère de Routh. Pour un système bouclé, ce critère utilise le dénominateur de la BF. Ce critère n'étant pas au programme, on pourra rechercher dans la littérature des articles s'y référant si nécessaire. 

\subsubsection{Critère graphique de stabilité : le critère du Revers}
 On a vu que l'équation caractéristique était de la forme $1+H_{BO}(p)=0$. Ainsi, Pour cela revient à résoudre l'équation $H_{BO}(p)=-1$. Ainsi dans le plan complexe, le point $(-1;0)$ permet d'avoir une information sur la stabilité. En terme de module et de phase, ce nombre complexe a un module de 1 (gain de 1) et une phase de $-180\degres$.
 
\begin{resultat}
Le système en boucle ouverte étant asymptotiquement stable (ou juste stable), le système en boucle fermée est asymptotiquement stable si et seulement si, \textbf{en boucle ouverte, on a} :
$$
\left. G_{\text{dB}} \right|_{\omega=\omega(-180\degres)}<0_{\text{dB}} 
\quad
\text{et}
\quad
\left. \varphi \right|_{\omega=\omega(0\text{dB})}>-180\degres.
$$
\end{resultat}
 
 
\begin{center}
\includegraphics[width=.9\linewidth]{images/marges}
\end{center} 
\subsubsection{Vers le système réel...}

Le résultat donné ci-dessus est un résultat théorique dans le sens ou le diagramme de Bode de la boucle ouverte du système réel aura un écart avec le diagramme de Bode du système modélisé. 


\begin{resultat}[Marges]
Pour tenir compte des écarts entre le modèle et le système réel, on est amené à définir une marge de gain et une marge de phase. Cela signifie que dans l'étude des systèmes asservis, on considèrera, dans le cas général que le système est stable si :
\begin{itemize}
\item la marge de gain est supérieure à \SI{10}{dB};
\item la marge de phase est supérieure à $45\degres$.
\end{itemize}
\end{resultat}

\begin{minipage}[c]{.58\linewidth}
\begin{defi}[Marge de phase]

La marge de phase est définie telle que $M_\varphi= \SI{180}{\degree} + \arg\left(\text{FTBO}(j\omega_{co})\right)$ où $\omega_{co}$ est la pulsation de coupure pour laquelle $|\text{FTBO}\left(j\omega_{co}\right)|=\SI{0}{dB}$.
\end{defi}

\begin{defi}[Marge de gain]
La marge de gain est définie telle
que $M_G = -20\log|\text{FTBO} (j\omega_{\varphi 180})|$
où $\omega_{\varphi 180}$ est la pulsation pour laquelle $\arg(\text{FTBO}(j\omega_{ \varphi 180}))= -\SI{180}{\degres}$.
\end{defi}
\end{minipage}\hfill
\begin{minipage}[c]{.4\linewidth}
\begin{center}
\includegraphics[width=.8\linewidth]{images/marges_fm}
\end{center}
\end{minipage}


 
La marge de gain permet compte de tenir compte de variations de gain de la boucle ouverte. 

De même, la marge de phase permet de tenir compte de variation de phase (retard ou déphasage non modélisés). 

La nécessité d'avoir recours à des marges de stabilité apparaît notamment lorsque : 
\begin{itemize}
\item la simplification du modèle amène à considérer uniquement les pôles dominant, 
\item le modèle ne prend pas en compte la dynamique de certains composants du système;
\item le système n'est pas invariant au cours du temps;
\item on s'éloigne de la zone de fonctionnement linéaire;
\item certaines non linéarités sont ignorées.
\end{itemize}



%%%%%%%%%%%%%%%%%%%%%%ùùù
\section{Rapidité des systèmes}
\subsection{Rappel : critère de rapidité dans le domaine temporel}

\subsubsection{Temps de réponse à 5\%}

\begin{methode}
[Détermination du temps de réponse à $n\%$] (En pratique $n=5$).\\

\begin{enumerate}
\item Tracer sur le même graphe la consigne $e(t)$ et la réponse du système
$s(t)$.
\item Tracer la droite correspondant à la valeur asymptotique de $s(t)$.
\item Tracer la bande correspondant à une variation de $\pm n\%$ de la valeur
asymptotique.
\item Relever la dernière valeur à partir de laquelle $s(t)$ coupe la bande et
n'en sort plus.
\end{enumerate}
\end{methode}

\begin{center}
%\includestandalone{images/perf}
\includegraphics{images/perf}
\end{center}

\begin{resultat}
Plus le temps de réponse à 5\% d'un système est petit, plus le régime transitoire disparaît rapidement. 
\end{resultat}

\begin{exemple}
Donner le temps de réponse à 5\% de la réponse à un échelon donné dans la figure suivante. 


\begin{minipage}[c]{.5\linewidth}
\begin{center}
%\includestandalone{images/perf}
\includegraphics[width=8cm]{images/tr.jpg}
\end{center}
\end{minipage} \hfill
\begin{minipage}[c]{.4\linewidth}
Les pièges du temps de réponse à 5\% :
\begin{itemize}
\item le temps de réponse à 5\% se mesure à plus ou moins 5\% de la sortie (et pas de l'entrée). Ainsi, si le système est stable, le temps de réponse n'est \textbf{jamais l'infini};
\item si le signal ne part pas de 0 (en ordonnée), il faut réaliser la bande à $S_0+\Delta s \pm 0.05\Delta s$;
\item si le signal ne part pas de 0 (en abscisse), il faut tenir compte du décalage des temps.
\end{itemize}
\end{minipage} 
\end{exemple}

\subsubsection{Temps de montée}


\begin{minipage}[c]{.48\linewidth}
Pour caractériser la rapidité d'un système, on peut aussi utiliser le temps de montée. Il s'agit du temps nécessaire pour passer de 10\% à 90\% de la valeur finale. Ce temps de montée caractériser la << vivacité >> d'un système. 
\end{minipage} \hfill
\begin{minipage}[c]{.4\linewidth}
\begin{center}
%\includestandalone{images/perf}
\includegraphics[width=\linewidth]{images/tm}
\end{center}
\end{minipage} 
\subsection{Rapidité des systèmes d'ordre 1 et d'ordre 2}
\subsubsection{Systèmes d'ordre 1}
Pour un système du premier ordre, le temps de réponse à 5\% est donné par $3\tau$.
\begin{resultat}
Pour un système du premier ordre, plus la constante de temps est petite, plus le système est rapide.
\end{resultat}

Soit un système du premier ordre bouclé avec un retour unitaire. L'expression de la FTBF est donnée par $FTBF(p)=\dfrac{K}{1+\tau p + K}$. La constante de temps est alors $\tau_{\text{BF}}=\dfrac{\tau}{1+K}$. 

\begin{resultat}
Pour un système du premier ordre bouclé (avec un retour unitaire), plus le gain statique est grand, plus le système est rapide. 
\end{resultat}

\subsubsection{Systèmes d'ordre 2}

\noindent\begin{minipage}[c]{.5\linewidth}
\begin{resultat}
Pour un système du second, à $\xi$ constant, plus la pulsation propre est grande, plus le système est rapide. 
\end{resultat} 



Soit un système du deuxième ordre bouclé avec un retour unitaire. En déterminant les caractéristiques de la FTBF, on obtient $K_{\text{BF}}=\dfrac{K}{1+K}$, $\omega_{\text{BF}}=\omega_0\sqrt{1+K}$, $\xi_{\text{BF}}=\dfrac{\xi}{1+K}$.


\begin{resultat}
\begin{itemize}
\item L'augmentation du gain de FTBO augmente la pulsation de la FTBF. 
\item L'augmentation du gain de FTBO diminue le coefficient d'amortissement. Suivant la valeur de $\xi{\text{BF}}$ le système peut devenir plus ou moins rapide.  
\end{itemize}
\end{resultat}

\end{minipage}\hfill
\begin{minipage}[c]{.47\linewidth}
\begin{center}
\includegraphics[width=\linewidth]{images/image7}
\end{center}
\end{minipage}

\subsection{Résultats dans le diagramme de Bode}


\noindent\begin{minipage}[c]{.48\linewidth}
\begin{resultat}
Plus la bande passante d'un système est élevée, plus le système est rapide.
\end{resultat}

\begin{center}
\includegraphics[height=4.5cm]{images/bandepassante}
\end{center}

\end{minipage} \hfill
\begin{minipage}[c]{.48\linewidth}
\begin{resultat}
Plus la pulsation de coupure à \SI{0}{dB} de la boucle ouverte est grande, plus le système asservi est rapide.
\end{resultat}


\begin{center}
\includegraphics[height=4.5cm]{images/bobf}
\end{center}
\end{minipage}


%%%%%%%%%%%%%%%% CHAP 3
\section{Précision des systèmes}
\subsection{Système non perturbé}
\begin{defi}
La précision est l'écart entre la valeur de consigne et la valeur de la sortie. Pour caractériser la précision d'un système, on s'intéresse généralement à l'écart en régime permanent.

Attention à bien s'assurer que, lors d'une mesure expérimentale par exemple, les grandeurs de consigne et de sortie sont bien de la même unité (et qualifient bien la même grandeur physique).

\vspace{.2cm}

\noindent \begin{minipage}[c]{.6\linewidth}
Pour un système non perturbé dont le schéma-blocs est celui donné ci-contre, on caractérise l'écart en régime permanent par :
$$
\varepsilon_{\text{permanent}}=\lim\limits_{t\to +\infty} \varepsilon(t)
\quad
\Longleftrightarrow 
\quad
\varepsilon_{\text{permanent}}=\lim\limits_{p\to 0} p\varepsilon(p)
$$
\end{minipage}
\hspace{.5cm}
\begin{minipage}[c]{.25\linewidth}
\includestandalone{images/Schema_1_entree_F_R}
\end{minipage}

\end{defi}

\begin{defi}
Un système est précis pour une entrée lorsque $\varepsilon_{\text{permanent}}=0$.
\end{defi}

\begin{defi} \~\\
Le nom de l'écart dépend de l'entrée avec lequel le système est sollicité : 
\begin{itemize}
\item écart statique, système sollicité par une entrée échelon : $e(t)=E_0$ et $E(p)=\dfrac{E_0}{p}$;
\item écart dynamique (en vitesse ou en poursuite), système sollicité par une rampe : $e(t)=Vt$ et $E(p)=\dfrac{V}{p^2}$;
\item écart en accélération : système sollicité par une parabole, $e(t)=At^2$ et $E(p)=\dfrac{A}{p^3}$.
\end{itemize}
\end{defi}

\begin{center}
\includegraphics[width=.9\linewidth]{images/fig_erreur}
\end{center}

\subsubsection*{Petit développement ...}

Calculons l'écart statique pour le système précédent. On a : $\varepsilon(p)=E(p)-R(p)=E(p)-\varepsilon(p) F(p) G(p)$. En conséquences,
$\varepsilon(p)=E(p)-\varepsilon(p) F(p) G(p) 
\Longleftrightarrow \varepsilon(p)\left( 1+F(p) G(p) \right) =E(p) 
\Longleftrightarrow   \varepsilon(p) =\dfrac{E(p)}{1+F(p) G(p)}$.

\begin{resultat}
$$\varepsilon(p) =\dfrac{E(p)}{1+FTBO(p)}$$
\end{resultat}


\subsubsection*{Poursuivons ...}
On a $FTBO(p)=\dfrac{K_{BO}\left(1+a_1p+...+a_mp^m \right)}{p^{\alpha}\left(1+b_1p+...+b_np^n\right)}$ avec $m< n$.
\subsubsection*{FTBO de classe nulle}

\begin{itemize}
\item Pour une entrée échelon : 
$\varepsilon_{\text{permanent}}=\lim\limits_{p\to 0} p\dfrac{E_0}{p}\dfrac{1}{1+FTBO(p)} 
= \dfrac{E_0}{1+K_{BO}}$.
\item Pour une entrée de type rampe : 
$\varepsilon_{\text{permanent}}=\lim\limits_{p\to 0} p\dfrac{V}{p^2}\dfrac{1}{1+FTBO(p)} 
=+\infty$.
\item Pour une entrée de type parabole : 
$\varepsilon_{\text{permanent}}=\lim\limits_{p\to 0} p\dfrac{A}{p^3}\dfrac{1}{1+FTBO(p)} 
=+\infty$.
\end{itemize}

\subsubsection*{FTBO de classe 1}

\begin{itemize}
\item Pour une entrée échelon : 
$\varepsilon_{\text{permanent}}=\lim\limits_{p\to 0} p\dfrac{E_0}{p}\dfrac{1}{1+\dfrac{K_{BO}\left(1+a_1p+...+a_mp^m \right)}{p\left(1+b_1p+...+b_np^n\right)}} 
= 0$.
\item Pour une entrée de type rampe : 
$\varepsilon_{\text{permanent}}=\lim\limits_{p\to 0} p\dfrac{V}{p^2}\dfrac{1}{1+\dfrac{K_{BO}\left(1+a_1p+...+a_mp^m \right)}{p\left(1+b_1p+...+b_np^n\right)}} 
=\dfrac{V}{K_{BO}}$.
\item Pour une entrée de type parabole : 
$\varepsilon_{\text{permanent}}=\lim\limits_{p\to 0} p\dfrac{A}{p^3}\dfrac{1}{1+\dfrac{K_{BO}\left(1+a_1p+...+a_mp^m \right)}{p\left(1+b_1p+...+b_np^n\right)}} 
=+\infty$.
\end{itemize}

\subsubsection*{FTBO de classe 2}

\begin{itemize}
\item Pour une entrée échelon : 
$\varepsilon_{\text{permanent}}=\lim\limits_{p\to 0} p\dfrac{E_0}{p}\dfrac{1}{1+\dfrac{K_{BO}\left(1+a_1p+...+a_mp^m \right)}{p^{2}\left(1+b_1p+...+b_np^n\right)}} 
= 0$.
\item Pour une entrée de type rampe : 
$\varepsilon_{\text{permanent}}=\lim\limits_{p\to 0} p\dfrac{V}{p^2}\dfrac{1}{1+\dfrac{K_{BO}\left(1+a_1p+...+a_mp^m \right)}{p^{2}\left(1+b_1p+...+b_np^n\right)}} 
=0$.
\item Pour une entrée de type parabole : 
$\varepsilon_{\text{permanent}}=\lim\limits_{p\to 0} p\dfrac{A}{p^3}\dfrac{1}{1+\dfrac{K_{BO}\left(1+a_1p+...+a_mp^m \right)}{p^{2}\left(1+b_1p+...+b_np^n\right)}} 
=\dfrac{A}{K_{BO}}$.
\end{itemize}


\begin{resultat} ~\\

\begin{center}
\begin{tabular}{|c|c|c|c|}
\hline 
Classe & Consigne échelon & Consigne en rampe & Consigne parabolique \\
& $e(t)=E_0$ & $e(t)=V t $ & $e(t)=At^2$ \\ 
& $E(p)=\dfrac{E_0}{p}$ & $E(p)=\dfrac{V}{p^2}$ & $E(p)=\dfrac{A}{p^3}$ \\ 
\hline \hline 
&&&\\
0 & $\varepsilon_S = \dfrac{E_0}{1+K_{BO}} $ & $\varepsilon_V = +\infty$ & $\varepsilon_A = +\infty$ \\
&&&\\
\hline 
&&&\\
1 & $\varepsilon_S = 0$ & $\varepsilon_V = \dfrac{V}{K_{BO}} $ & $\varepsilon_A = +\infty$ \\
&&&\\
\hline 
&&&\\
2 & $\varepsilon_S = 0 $ & $\varepsilon_V = 0$ & $\varepsilon_A = \dfrac{A}{K_{BO}}$ \\
&&&\\
\hline 
\end{tabular}
\end{center}

\begin{rem}
L'écart statique est nul si la boucle ouverte comprend au moins une intégration. À défaut, l'augmentation du gain statique de la boucle ouverte provoque une amélioration de la précision.
\end{rem}

\end{resultat}




%\begin{methode}[Détermination de l'erreur pour un système non perturbé]
%
%\end{methode}

%\begin{methode}[Détermination de l'erreur pour un système perturbé]
%
%\end{methode}
%
%\begin{resultat}
%Tableau...
%\end{resultat}

\subsection{Système perturbé}
Soit le schéma-blocs suivant : 
\begin{center}
\includestandalone{images/Schema2Entrees_2F_R}
\end{center}

\vspace{.25cm}

\textbf{L'écart est caractérisé par le soustracteur principal, c'est-à-dire celui situé le plus à gauche du schéma-blocs.}

\vspace{.25cm}

Par lecture directe, on a : 
$\varepsilon(p)
=E(p)-R(p)S(p)
=E(p)-R(p)\left(H_2(p) \left(P(p)+\varepsilon(p)H_1(p) \right)\right)$
$\Longleftrightarrow 
\varepsilon(p) =E(p)- R(p)H_2(p)P(p)-R(p)H_1(p)H_2(p)\varepsilon(p) $
$\Longleftrightarrow \varepsilon(p)\left( 1+R(p)H_1(p)H_2(p)\right) =E(p)- R(p)H_2(p)P(p)$
$\Longleftrightarrow \varepsilon(p) =\dfrac{E(p)}{1+R(p)H_1(p)H_2(p)}- \dfrac{R(p)H_2(p)}{1+R(p)H_1(p)H_2(p)}P(p)$.

On a donc :
$\varepsilon(p) =\underbrace{\dfrac{1}{1+\text{FTBO}(p)} E(p)}_{\text{\'Ecart vis-à-vis de la consigne}} - \underbrace{\dfrac{R(p)H_2(p)}{1+\text{FTBO}(p)}P(p)}_{\text{\'Ecart vis-à-vis de la perturbation}}$.

Notons $H_1(p)=\dfrac{K_1}{p^\alpha_1}\dfrac{N_1(p)}{D_1(p)}$ (avec $N_1(0)=1$ et $D_1(0)=1$) et $H_2(p)=\dfrac{K_2}{p^\alpha_2}\dfrac{N_2(p)}{D_2(p)}$ (avec $N_2(0)=1$ et $D_2(0)=1$).

Par ailleurs 
$H_{\text{bo}}(p)=H_1(p)H_2(p)R(p)
=\dfrac{H_{\text{bo}}}{p^{\alpha}}\dfrac{N(p)}{D(p)}$.

On peut calculer l'écart vis-à-vis de la perturbation : 
$\varepsilon_{\text{perturbation}}
= \lim\limits_{t\to\infty}\varepsilon(t)
= \lim\limits_{p\to 0}p \varepsilon(p) 
= \lim\limits_{p\to 0} \dfrac{K_2 p^{\alpha_1+1}}{p^{\alpha_1+\alpha_2}+K_1K_2}P(p) $.

%$S(p)=H_2(p)\left(P(p)+H_1(p)\left(E(p)-R(p)S(p)\right) \right)$
%$=P(p)H_2(p)+H_1(p)H_2(p)E(p)-H_1(p)H_2(p)R(p)S(p)$ 
%$\Leftrightarrow S(p)\left( 1+H_1(p)H_2(p)R(p)\right)=P(p)H_2(p)+H_1(p)H_2(p)E(p)$
%$\Leftrightarrow S(p)=P(p)\dfrac{H_2(p)}{1+H_1(p)H_2(p)R(p)}+\dfrac{H_1(p)H_2(p)}{1+H_1(p)H_2(p)R(p)}E(p)$.

%On a donc $S(p)=P(p)\dfrac{H_2(p)}{1+\text{FTBO}(p)}+\dfrac{H_1(p)H_2(p)}{1+%\text{FTBO}(p)}E(p)$.
\begin{center}
\begin{tabular}{|c|c|c|}
\hline
Cas & Classe du système & Perturbation en échelon $P(p)=\dfrac{P_0}{p}$ \\ \hline
1 & $\alpha_1\geq 1$ & $\varepsilon_{\text{perturbation}}=0$ \\ \hline
2 & $\alpha_1=0$ et $\alpha_2=0$ & $\varepsilon_{\text{perturbation}}=\dfrac{K_2}{1+K_1K_2}P_0$ \\ \hline
3 & $\alpha_1=0$ et $\alpha_2\geq 1$ & $\varepsilon_{\text{perturbation}}=\dfrac{P_0}{K_1}$ \\ \hline
\end{tabular}
\end{center}

\begin{resultat}
Il faut au moins un intégrateur en amont d'une perturbation constante pour
annuler l'écart vis-à-vis de cette perturbation. Un intégrateur placé en aval n'a aucune
influence.

Quand ce n'est pas le cas, un gain $K_1$ important en amont de la perturbation réduit toujours
l'écart vis-à-vis de cette perturbation.
\end{resultat}


\begin{center}
\includegraphics[width=.7\linewidth]{images/fig_erreur_02}
\end{center}





\begin{thebibliography}{2}
   \bibitem[1]{ref1} Frédéric Mazet, {\it Cours d'automatique de deuxième année, Lycée Dumont Durville, Toulon.}
      \bibitem[2]{ref2} Florestan Mathurin, {\it Stabilité des SLCI, Lycée Bellevue, Toulouse, \url{http://florestan.mathurin.free.fr/}.}



\end{thebibliography}

\end{document}



