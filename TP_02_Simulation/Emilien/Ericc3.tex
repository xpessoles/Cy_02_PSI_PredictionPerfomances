\documentclass[10pt,fleqn]{article} % Default font size and left-justified equations
\usepackage[%
    pdftitle={Modélisation SLCI : Précision des systèmes},
    pdfauthor={Xavier Pessoles}]{hyperref}

\input{style/new_style}
\input{style/macros_SII}

\fichetrue
\fichefalse

\proftrue
%\proffalse

%\tdtrue
\tdfalse

\courstrue
%\coursfalse



% -------------------------------------
% Déclaration des titres
% -------------------------------------

\def\discipline{Sciences \\Industrielles de \\ l'Ingénieur}
\def\xxtete{Sciences Industrielles de l'Ingénieur}

\def\classe{\textsf{PSI$\star$ -- MP}}
\def\xxnumpartie{Cycle 02}
\def\xxpartie{Modéliser les systèmes asservis dans le but de prévoir leur comportement}

\def\xxnumchapitre{Chapitre 3 \vspace{.2cm}}
\def\xxchapitre{\hspace{.12cm} Précision des systèmes}

\def\xxposongletx{2}
\def\xxposonglettext{1.45}
\def\xxposonglety{19}%16

\def\xxonglet{Cycle 02}

\def\xxactivite{Cours}
\def\xxauteur{\textsl{Xavier Pessoles}}

\def\xxcompetences{%
\textsl{%
\textbf{Savoirs et compétences :}\\
\begin{itemize}[label=\ding{112},font=\color{ocre}] 
\item \textit{Res2.C10 : } précision des SLCI : erreur en régime permanent
\item \textit{Res2.C11 : } précision des SLCI : influence de la classe de la fonction de transfert en boucle ouverte
\item \textit{Res2.C10.SF1 : } déterminer l’erreur en régime permanent vis-à-vis d’une entrée en échelon ou en rampe (consigne ou perturbation)
\item \textit{Res2.C11.SF1 : } relier la précision aux caractéristiques fréquentielles
\end{itemize}
}}

	
		
	
		


\def\xxfigures{
%\includegraphics[width=1.4\textwidth]{images/matlab}%images/prot_01
%\\
%\textit{Modèle du pilote hydraulique avec pilotage interactif.}
}%figues de la page de garde

\def\xxpied{%
Cycle 02 -- Modéliser les SLCI dans le but de prévoir leur comportement\\
Chapitre 3 -- \xxactivite%
}

\setcounter{secnumdepth}{5}
%---------------------------------------------------------------------------


\begin{document}
\chapterimage{png/Fond_SLCI}
\input{style/new_pagegarde}
\setlength{\columnseprule}{.1pt}

\vspace{2cm}
\pagestyle{fancy}
\thispagestyle{plain}
\section{Système non perturbé}
\begin{defi}~\\


La précision est l'écart entre la valeur de consigne et la valeur de la sortie. Pour caractériser la précision d'un système, on s'intéresse généralement à l'écart en régime permanent.

Attention à bien s'assurer que, lors d'une mesure expérimentale par exemple, les grandeurs de consigne et de sortie sont bien de la même unité (et qualifient bien la même grandeur physique).

\vspace{.2cm}

\noindent \begin{minipage}[c]{.6\linewidth}
Pour un système non perturbé dont le schéma-blocs est celui donné ci-contre, on caractérise l'écart en régime permanent par :
$$
\varepsilon_{\text{permanent}}=\lim\limits_{t\to +\infty} \varepsilon(t)
\quad
\Longleftrightarrow 
\quad
\varepsilon_{\text{permanent}}=\lim\limits_{p\to 0} p\varepsilon(p)
$$
\end{minipage}
\hspace{.5cm}
\begin{minipage}[c]{.25\linewidth}
\includestandalone{images/Schema_1_entree_F_R}
\end{minipage}

\end{defi}



\begin{defi} \~\\
Le nom de l'écart dépend de l'entrée avec lequel le système est sollicité : 
\begin{itemize}
\item écart statique : système sollicité par une entrée échelon -- $e(t)=E_0$ et $E(p)=\dfrac{E_0}{p}$;
\item écart dynamique (en vitesse ou en poursuite) : système sollicité par une rampe -- $e(t)=Vt$ et $E(p)=\dfrac{V}{p^2}$;
\item écart en accélération : système sollicité par une parabole  -- $e(t)=At^2$ et $E(p)=\dfrac{A}{p^2}$.
\end{itemize}
\end{defi}


\subsection*{Petit développement ...}

Calculons l'écart statique pour le système précédent. On a : $\varepsilon(p)=E(p)-R(p)=E(p)-\varepsilon(p) F(p) G(p)$. En conséquences,
$\varepsilon(p)=E(p)-\varepsilon(p) F(p) G(p) 
\Longleftrightarrow \varepsilon(p)\left( 1+F(p) G(p) \right) =E(p) 
\Longleftrightarrow   \varepsilon(p) =\dfrac{E(p)}{1+F(p) G(p)}$.

\begin{resultat} ~\\

$$\varepsilon(p) =\dfrac{E(p)}{1+FTBO(p)}$$
\end{resultat}


\subsection*{Poursuivons ...}
On a $FTBO(p)=\dfrac{K_{BO}\left(1+a_1p+...+a_mp^m \right)}{p^{\alpha}\left(1+b_1p+...+b_np^n\right)}$ avec $m< n$.
\subsubsection*{FTBO de classe nulle}

\begin{itemize}
\item Pour une entrée échelon : 
$\varepsilon_{\text{permanent}}=\lim\limits_{p\to 0} p\dfrac{E_0}{p}\dfrac{1}{1+FTBO(p)} 
= \dfrac{E_0}{1+K_{BO}}$.
\item Pour une entrée de type rampe : 
$\varepsilon_{\text{permanent}}=\lim\limits_{p\to 0} p\dfrac{V}{p^2}\dfrac{1}{1+FTBO(p)} 
=+\infty$.
\item Pour une entrée de type parabole : 
$\varepsilon_{\text{permanent}}=\lim\limits_{p\to 0} p\dfrac{A}{p^3}\dfrac{1}{1+FTBO(p)} 
=+\infty$.
\end{itemize}

\subsubsection*{FTBO de classe 1}

\begin{itemize}
\item Pour une entrée échelon : 
$\varepsilon_{\text{permanent}}=\lim\limits_{p\to 0} p\dfrac{E_0}{p}\dfrac{1}{1+\dfrac{K_{BO}\left(1+a_1p+...+a_mp^m \right)}{p\left(1+b_1p+...+b_np^n\right)}} 
= 0$.
\item Pour une entrée de type rampe : 
$\varepsilon_{\text{permanent}}=\lim\limits_{p\to 0} p\dfrac{V}{p^2}\dfrac{1}{1+\dfrac{K_{BO}\left(1+a_1p+...+a_mp^m \right)}{p\left(1+b_1p+...+b_np^n\right)}} 
=\dfrac{V}{K_{BO}}$.
\item Pour une entrée de type parabole : 
$\varepsilon_{\text{permanent}}=\lim\limits_{p\to 0} p\dfrac{A}{p^3}\dfrac{1}{1+\dfrac{K_{BO}\left(1+a_1p+...+a_mp^m \right)}{p\left(1+b_1p+...+b_np^n\right)}} 
=+\infty$.
\end{itemize}

\subsubsection*{FTBO de classe 2}

\begin{itemize}
\item Pour une entrée échelon : 
$\varepsilon_{\text{permanent}}=\lim\limits_{p\to 0} p\dfrac{E_0}{p}\dfrac{1}{1+\dfrac{K_{BO}\left(1+a_1p+...+a_mp^m \right)}{p^{2}\left(1+b_1p+...+b_np^n\right)}} 
= 0$.
\item Pour une entrée de type rampe : 
$\varepsilon_{\text{permanent}}=\lim\limits_{p\to 0} p\dfrac{V}{p^2}\dfrac{1}{1+\dfrac{K_{BO}\left(1+a_1p+...+a_mp^m \right)}{p^{2}\left(1+b_1p+...+b_np^n\right)}} 
=0$.
\item Pour une entrée de type parabole : 
$\varepsilon_{\text{permanent}}=\lim\limits_{p\to 0} p\dfrac{A}{p^3}\dfrac{1}{1+\dfrac{K_{BO}\left(1+a_1p+...+a_mp^m \right)}{p^{2}\left(1+b_1p+...+b_np^n\right)}} 
=\dfrac{A}{K_{BO}}$.
\end{itemize}


\begin{resultat} ~\\

\begin{center}
\begin{tabular}{|c|c|c|c|}
\hline 
Classe & Consigne échelon & Consigne en rampe & Consigne parabolique \\
& $e(t)=E_0$ & $e(t)=V t $ & $e(t)=At^2$ \\ 
& $E(p)=\dfrac{E_0}{p}$ & $E(p)=\dfrac{V}{p^2}$ & $E(p)=\dfrac{A}{p^3}$ \\ 
\hline \hline 
&&&\\
0 & $\varepsilon_S = \dfrac{E_0}{1+K_{BO}} $ & $\varepsilon_V = +\infty$ & $\varepsilon_A = +\infty$ \\
&&&\\
\hline 
&&&\\
1 & $\varepsilon_S = 0$ & $\varepsilon_V = \dfrac{V}{K_{BO}} $ & $\varepsilon_A = +\infty$ \\
&&&\\
\hline 
&&&\\
2 & $\varepsilon_S = 0 $ & $\varepsilon_V = 0$ & $\varepsilon_A = \dfrac{A}{K_{BO}}$ \\
&&&\\
\hline 
\end{tabular}
\end{center}

\begin{rem}
L'écart statique est nul si la boucle ouverte comprend au moins une intégration. À défaut, l'augmentation du gain statique de la boucle ouverte provoque une amélioration de la précision.
\end{rem}

\end{resultat}




%\begin{methode}[Détermination de l'erreur pour un système non perturbé]
%
%\end{methode}

\begin{methode}[Détermination de l'erreur pour un système perturbé]

\end{methode}

\begin{resultat}
Tableau...
\end{resultat}

\section{Système perturbé}

\section{Précision et réponse fréquentielle}


\begin{thebibliography}{2}
   \bibitem[1]{ref1} Frédéric Mazet, {\it Cours d'automatique de deuxième année, Lycée Dumont Durville, Toulon.}
      \bibitem[2]{ref2} Florestan Mathurin, {\it Stabilité des SLCI, Lycée Bellevue, Toulouse, \url{http://florestan.mathurin.free.fr/}.}



\end{thebibliography}

\end{document}



