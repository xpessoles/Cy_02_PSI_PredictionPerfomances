\documentclass[10pt,fleqn]{article} % Default font size and left-justified equations
\usepackage[%
    pdftitle={Modélisation SLCI : Précision des systèmes},
    pdfauthor={Xavier Pessoles}]{hyperref}

\input{style/new_style}
\input{style/macros_SII}

\fichetrue
\fichefalse

\proftrue
%\proffalse

%\tdtrue
\tdfalse

\courstrue
%\coursfalse



% -------------------------------------
% Déclaration des titres
% -------------------------------------

\def\discipline{Sciences \\Industrielles de \\ l'Ingénieur}
\def\xxtete{Sciences Industrielles de l'Ingénieur}

\def\classe{PSI$\star$ -- MP}
\def\xxnumpartie{Cycle 02}
\def\xxpartie{Modéliser les systèmes asservis dans le but de prévoir leur comportement}

\def\xxnumchapitre{Chapitre 3 \vspace{.2cm}}
\def\xxchapitre{\hspace{.12cm} Précision des systèmes}

\def\xxposongletx{2}
\def\xxposonglettext{1.45}
\def\xxposonglety{19}%16

\def\xxonglet{Cycle 02}

\def\xxactivite{Cours}
\def\xxauteur{\textsl{Xavier Pessoles}}

\def\xxcompetences{%
\textsl{%
\textbf{Savoirs et compétences :}\\
\begin{itemize}[label=\ding{112},font=\color{ocre}] 
\item \textit{Res2.C10 : } précision des SLCI : erreur en régime permanent
\item \textit{Res2.C11 : } précision des SLCI : influence de la classe de la fonction de transfert en boucle ouverte
\item \textit{Res2.C10.SF1 : } déterminer l’erreur en régime permanent vis-à-vis d’une entrée en échelon ou en rampe (consigne ou perturbation)
\item \textit{Res2.C11.SF1 : } relier la précision aux caractéristiques fréquentielles
\end{itemize}
}}

	
		
	
		


\def\xxfigures{
%\includegraphics[width=1.4\textwidth]{images/matlab}%images/prot_01
%\\
%\textit{Modèle du pilote hydraulique avec pilotage interactif.}
}%figues de la page de garde

\def\xxpied{%
Cycle 02 -- Modéliser les SLCI dans le but de prévoir leur comportement\\
Chapitre 3 -- \xxactivite%
}

\setcounter{secnumdepth}{5}
%---------------------------------------------------------------------------


\begin{document}
\chapterimage{png/Fond_SLCI}
\input{style/new_pagegarde}
\setlength{\columnseprule}{.1pt}

\vspace{2cm}
\pagestyle{fancy}
\thispagestyle{plain}
\section{Définition}
\begin{defi}
écart erreur
\end{defi}

\begin{defi}
Écart statique, écart dynamique (en vitesse ou en poursuite), écart en accélération
\end{defi}

\begin{methode}[Détermination de l'erreur pour un système non perturbé]

\end{methode}

\begin{methode}[Détermination de l'erreur pour un système perturbé]

\end{methode}

\begin{resultat}
Tableau...
\end{resultat}
\section{Précision et réponse fréquentielle}


\begin{thebibliography}{2}
   \bibitem[1]{ref1} Frédéric Mazet, {\it Cours d'automatique de deuxième année, Lycée Dumont Durville, Toulon.}
      \bibitem[2]{ref2} Florestan Mathurin, {\it Stabilité des SLCI, Lycée Bellevue, Toulouse, \url{http://florestan.mathurin.free.fr/}.}



\end{thebibliography}

\end{document}



