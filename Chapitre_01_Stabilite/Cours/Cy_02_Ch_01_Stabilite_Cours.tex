\documentclass[10pt,fleqn]{article} % Default font size and left-justified equations
\usepackage[%
    pdftitle={Modélisation systèmes multiphysiques : Modélisation linéaire et non linéaire},
    pdfauthor={Xavier Pessoles}]{hyperref}

\input{style/new_style}
\input{style/macros_SII}

\fichetrue
\fichefalse

\proftrue
%\proffalse

%\tdtrue
\tdfalse

\courstrue
%\coursfalse



% -------------------------------------
% Déclaration des titres
% -------------------------------------

\def\discipline{Sciences \\Industrielles de \\ l'Ingénieur}
\def\xxtete{Sciences Industrielles de l'Ingénieur}

\def\classe{\textsf{Cy 02}}
\def\xxnumpartie{Cycle 02}
\def\xxpartie{Modéliser les systèmes asservis dans le but de prévoir leur comportement}

\def\xxnumchapitre{Chapitre 1 \vspace{.2cm}}
\def\xxchapitre{\hspace{.12cm} Stabilité des systèmes}

\def\xxposongletx{2}
\def\xxposonglettext{1.45}
\def\xxposonglety{19}%16

\def\xxonglet{Cycle 02}

\def\xxactivite{Cours}
\def\xxauteur{\textsl{Xavier Pessoles}}

\def\xxcompetences{%
\textsl{%
\textbf{Savoirs et compétences :}\\
\begin{itemize}[label=\ding{112},font=\color{ocre}] 
\item \textit{Mod3.C2 : } pôles dominants et réduction de l’ordre du modèle : principe, justification
\item \textit{Res2.C4 : } stabilité des SLCI : définition entrée bornée -- sortie bornée (EB -- SB)	
\item \textit{Res2.C5 : } stabilité des SLCI : équation caractéristique	
\item \textit{Res2.C6 : } stabilité des SLCI : position des pôles dans le plan complexe
\item \textit{Res2.C7 : } stabilité des SLCI : marges de stabilité (de gain et de phase)
\end{itemize}
}}

	
		
	
		


\def\xxfigures{
%\includegraphics[width=1.4\textwidth]{images/matlab}%images/prot_01
%\\
%\textit{Modèle du pilote hydraulique avec pilotage interactif.}
}%figues de la page de garde

\def\xxpied{%
Cycle 02 -- Modéliser les SLCI dans le but de prévoir leur comportement\\
Chapitre 1 -- \xxactivite%
}

\setcounter{secnumdepth}{5}
%---------------------------------------------------------------------------


\begin{document}
\chapterimage{png/Fond_SLCI}
\input{style/new_pagegarde}
\setlength{\columnseprule}{.1pt}

\vspace{2cm}
\pagestyle{fancy}
\thispagestyle{plain}
\section{Notion de stabilité}
\subsection{Représentation graphique}
\subsection{Premières définitions}
\begin{defi}[Définition intuitive]
Un système est asymptotiquement stable si et seulement si : 
\begin{itemize}
\item abandonné à lui-même à partir de conditions initiales quelconques il revient à son état d'équilibre;
\item son régime transitoire finit par disparaître;
\item sa sortie finit par ressembler à l'entrée;
\item sa réponse tend vers zéro au cours du temps.
\end{itemize}

\end{defi}

\begin{rem}
La stabilité d'un système \textbf{est indépendante} de la nature de l'entrée. Ainsi, l'étude de la stabilité peut se faire à partir d'une réponse impulsionnelle (entrée Dirac), indicielle (entrée échelon d'amplitude 1), d'une réponse harmonique (entrée sinusoïdale)...

Pour simplifier les calculs, une première approche pourra être d'utiliser la réponse impulsionnelle. 
\end{rem}
\begin{defi}
En conséquence, on peut considérer qu'un système est asymptotiquement stable si et seulement si sa réponse impulsionnelle tend vers zéro au cours du temps.
\end{defi}
\section{Étude des pôles de la fonction de transfert}

\begin{thebibliography}{2}
   \bibitem[1]{ref1} Frédéric Mazet, {\it Cours d'automatique de deuxième année, Lycée Dumont Durville, Toulon.}
      \bibitem[2]{ref2} Florestan Mathurin, {\it Stabilité des SLCI, Lycée Bellevue, Toulouse, \url{http://florestan.mathurin.free.fr/}.}


\end{thebibliography}

\end{document}



