\documentclass[10pt,fleqn]{article} % Default font size and left-justified equations
\usepackage[%
    pdftitle={Modélisation systèmes multiphysiques : Modélisation linéaire et non linéaire},
    pdfauthor={Xavier Pessoles}]{hyperref}

\input{style/new_style}
\input{style/macros_SII}

\fichetrue
\fichefalse

\proftrue
%\proffalse

%\tdtrue
\tdfalse

\courstrue
%\coursfalse



% -------------------------------------
% Déclaration des titres
% -------------------------------------

\def\discipline{Sciences \\Industrielles de \\ l'Ingénieur}
\def\xxtete{Sciences Industrielles de l'Ingénieur}

\def\classe{\textsf{Cy 02}}
\def\xxnumpartie{Cycle 02}
\def\xxpartie{Modéliser les systèmes asservis dans le but de prévoir leur comportement}

\def\xxnumchapitre{Chapitre 1 \vspace{.2cm}}
\def\xxchapitre{\hspace{.12cm} Stabilité des systèmes}

\def\xxposongletx{2}
\def\xxposonglettext{1.45}
\def\xxposonglety{19}%16

\def\xxonglet{Cycle 02}

\def\xxactivite{Cours}
\def\xxauteur{\textsl{Xavier Pessoles}}

\def\xxcompetences{%
\textsl{%
\textbf{Savoirs et compétences :}\\
\begin{itemize}[label=\ding{112},font=\color{ocre}] 
\item \textit{Mod3.C2 : } pôles dominants et réduction de l’ordre du modèle : principe, justification
\item \textit{Res2.C4 : } stabilité des SLCI : définition entrée bornée -- sortie bornée (EB -- SB)	
\item \textit{Res2.C5 : } stabilité des SLCI : équation caractéristique	
\item \textit{Res2.C6 : } stabilité des SLCI : position des pôles dans le plan complexe
\item \textit{Res2.C7 : } stabilité des SLCI : marges de stabilité (de gain et de phase)
\end{itemize}
}}

	
		
	
		


\def\xxfigures{
%\includegraphics[width=1.4\textwidth]{images/matlab}%images/prot_01
%\\
%\textit{Modèle du pilote hydraulique avec pilotage interactif.}
}%figues de la page de garde

\def\xxpied{%
Cycle 02 -- Modéliser les SLCI dans le but de prévoir leur comportement\\
Chapitre 1 -- \xxactivite%
}

\setcounter{secnumdepth}{5}
%---------------------------------------------------------------------------


\begin{document}
\chapterimage{png/Fond_SLCI}
\input{style/new_pagegarde}
\setlength{\columnseprule}{.1pt}

\vspace{2cm}
\pagestyle{fancy}
\thispagestyle{plain}
\section{Notion de stabilité}
\subsection{Représentation graphique}
\subsection{Premières définitions}
\begin{defi}[Définition intuitive]

Un système est asymptotiquement stable si et seulement si : 
\begin{itemize}
\item abandonné à lui-même à partir de conditions initiales quelconques il revient à son état d'équilibre;
\item son régime transitoire finit par disparaître;
\item sa sortie finit par ressembler à l'entrée;
\item sa réponse tend vers zéro au cours du temps.
\end{itemize}

\end{defi}

\begin{rem}
La stabilité d'un système \textbf{est indépendante} de la nature de l'entrée. Ainsi, l'étude de la stabilité peut se faire à partir d'une réponse impulsionnelle (entrée Dirac), indicielle (entrée échelon d'amplitude 1), d'une réponse harmonique (entrée sinusoïdale)...

Pour simplifier les calculs, une première approche pourra être d'utiliser la réponse impulsionnelle. 
\end{rem}
\begin{defi}
En conséquence, on peut considérer qu'un système est asymptotiquement stable si et seulement si sa réponse impulsionnelle tend vers zéro au cours du temps.
\end{defi}

\subsection{Étude des pôles de la fonction de transfert}
Dans le cas général la fonction de transfert d'un système peut se mettre sous la forme :
$$
H(p)=\dfrac{b_mp^m + b_{m-1}p^{m-1}+...+b_1p+b_0}{a_np^n + a_{n-1}p^{n-1}+...+a_1p+a_0} \quad \text{avec } n\geq m.
$$

Lors du calcul de la réponse temporelle en utilisant la transformée de Laplace inverse (quelle que soit l'entrée), la nature du régime transitoire ne dépend que des pôles $p_i$de la fonction de transfert (zéros du dénominateur).

En factorisant le numérateur et le dénominateur de $H(p)$ on peut alors retrouver une fonction de la forme  :
$$
H(p)=\dfrac{\left(p+ z_m\right)\cdot \left(p+ z_{m-1}\right)...}{\left(p+ p_n\right)\cdot \left(p+ p_{n-1}\right)...} \quad \text{avec } p_i,z_i\in \mathbb{C}.
$$

En passant dans le domaine temporel : 
\begin{itemize}
\item les pôles réels (de type $p=-a$) induisent des modes\footnote{mode : fonction temporelle associée à un pôle} du type $e^{-at}$;
\item les pôles complexes conjugués (de type $p=-a\pm j\omega$) induisent des modes du type 
$e^{-at} \sin \omega t$.
\end{itemize}

\textbf{On peut ainsi constater que si les pôles sont à partie réelle strictement négative, l'exponentielle décroissante permet de stabiliser la réponse temporelle.}

Ainsi, on peut observer la réponse temporelle des systèmes en fonction du positionnement des pôles dans le plan complexe. 

\begin{center}
\includegraphics[width=14cm]{images/poles_simple_double}

\textit{Représentation d'un système à pôle simple et à pôles conjugués dans le plan complexe -- Réponse indicielle}
\end{center}

\subsection{Position des pôles dans le plan complexe}
Par extension on peut observer dans le plan complexe les pôles de fonctions de transfert et leur indicielle associée.

\begin{center}
\includegraphics[width=\linewidth]{images/plan_complexe_fm}

\textit{Allure de la réponse à l’impulsion de Dirac selon la position des pôles de la FTBF d’un système \cite{2}.}
\end{center}

\begin{defi}[À retenir]
Un système est asymptotiquement stable si et seulement si tous les pôles de sa fonction de transfert sont à partie réelle strictement négative. 
\end{defi}

\begin{rem}On peut montrer que :
\begin{itemize}
\item \textbf{pour les systèmes d'ordre 1 et 2 :} le système est stable si tous les coefficients du dénominateur sont non nuls et de même signe;
\item \textbf{pour les systèmes d'ordre 3 :} de la forme $a_0+a_1p+a_2p^2+a_3p^3$ les coefficients doivent être strictement de même signe et $a_2 a_1 > a_3 a_0$.
\end{itemize}
\end{rem}

\subsection{Pôles dominants \cite{1}}
Lors de l’étude d’un système, on se contente en général de ne prendre en compte que les pôles les plus influents. Ces pôles sont appelés les pôles dominants. Pour un système asymptotiquement stable, ce sont ceux qui sont le plus proche de l’axe des imaginaires, puisque ce sont eux qui induisent des modes qui disparaissent dans le temps le plus lentement.

\section{Marges de stabilité}
\subsection{Lorsque la BO commence à pointer le bout de son nez...}


Soit le schéma-blocs suivant : 

\includestandalone{images/Schema_1_entree_F_R}

La fonction de transfert en boucle ouverte est donnée par $H_{BO}=\dfrac{R(p)}{\varepsilon(p)}=F(p)G(p)$. 

La fonction de transfert en boucle fermée est donnée par : $H_{BF}=\dfrac{S(p)}{E(p)}=\dfrac{F(p)}{1+F(p)G(p)}=\dfrac{F(p)}{1+H_{BO}(p)}$. 

\begin{defi}[Équation caractéristique]
Soit $H(p)=\dfrac{N(p)}{D(p)}$ une fonction de transfert. On appelle $D(p)=0$ l'équation caractéristique de la fonction de transfert. Ainsi les racines de $D(p)$ correspondent aux pôles de $H(p)$.

\textbf{Pour un système bouclé, l'équation caractéristique sera $1+H_{BO}(p)$.}

\end{defi}

\begin{thebibliography}{2}
   \bibitem[1]{ref1} Frédéric Mazet, {\it Cours d'automatique de deuxième année, Lycée Dumont Durville, Toulon.}
      \bibitem[2]{ref2} Florestan Mathurin, {\it Stabilité des SLCI, Lycée Bellevue, Toulouse, \url{http://florestan.mathurin.free.fr/}.}



\end{thebibliography}

\end{document}



