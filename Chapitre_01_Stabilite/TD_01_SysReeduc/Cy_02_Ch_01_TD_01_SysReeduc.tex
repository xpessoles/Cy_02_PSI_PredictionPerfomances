\documentclass[10pt,fleqn]{article} % Default font size and left-justified equations
\usepackage[%
    pdftitle={Modélisation systèmes multiphysiques : Modélisation linéaire et non linéaire},
    pdfauthor={Xavier Pessoles}]{hyperref}
    
\input{style/new_style}
\input{style/macros_SII}
\usepackage{multicol}
\usepackage{standalone}
\standaloneconfig{mode=buildnew}
\usepackage{siunitx}
\usepackage{wrapfig}
\fichetrue

%\fichefalse

\proftrue
\proffalse

\tdtrue
%\tdfalse

\courstrue
\coursfalse

\def\discipline{Sciences \\Industrielles de \\ l'Ingénieur}
\def\xxtete{Sciences Industrielles de l'Ingénieur}

\def\classe{PSI$\star$ -- MP}
\def\xxnumpartie{Cycle 02}
\def\xxpartie{Modéliser les systèmes asservis dans le but de prévoir leur comportement}

\def\xxnumchapitre{Chapitre 1 \vspace{.2cm}}
\def\xxchapitre{\hspace{.12cm} Stabilité des systèmes}


\def\xxtitreexo{\noindent Machine de rééducation SysReeduc}
\def\xxsourceexo{\hspace{.2cm} \footnotesize{CCP PSI 2012}}


\def\xxposongletx{2}
\def\xxposonglettext{1.45}
\def\xxposonglety{20}
%\def\xxonglet{Part. 1 -- Ch. 3}
\def\xxonglet{\textsf{Cycle 02}}

\def\xxactivite{TD 1}
\def\xxauteur{\textsl{Xavier Pessoles}}

\def\xxcompetences{%
\textsl{%
\footnotesize
\textbf{Savoirs et compétences :}\\
\begin{itemize}[label=\ding{112},font=\color{ocre}] 
%\item \textit{Mod3.C2 : } pôles dominants et réduction de l’ordre du modèle : principe, justification
%\item \textit{Res2.C4 : } stabilité des SLCI : définition entrée bornée -- sortie bornée (EB -- SB)	
%\item \textit{Res2.C5 : } stabilité des SLCI : équation caractéristique	
%\item \textit{Res2.C6 : } stabilité des SLCI : position des pôles dans le plan complexe
\item \textit{Res2.C7 : } stabilité des SLCI : marges de stabilité (de gain et de phase)
\end{itemize}
\normalsize
%Les sources sont associées par un \emph{hacheur série}. La détermination des grandeurs électriques associées à ce montage permet de conclure vis à vis du cahier des charges.
%\noindent \textbf{Résoudre :} à partir des modèles retenus :
%\begin{itemize}[label=\ding{112},font=\color{ocre}] 
%\item choisir une méthode de résolution analytique, graphique, numérique;
%\item mettre en \oe{}uvre une méthode de résolution.
%\end{itemize}
%\begin{itemize}[label=\ding{112},font=\color{ocre}] 
%\item \textit{Rés -- C1.1 :} Loi entrée sortie géométrique et cinématique -- Fermeture géométrique.
%\end{itemize}
%
%\noindent \textit{Mod2 -- C4.1 :} Représentation par schéma-blocs.
}}

\def\xxfigures{
\includegraphics[width=.6\linewidth]{images/fig_01}
}%figues de la page de garde


\def\xxpied{%
Cycle 02 -- Modéliser les SLCI -- Prédiction du comportement\\
Chapitre 1 -- \xxactivite%
}

\setcounter{secnumdepth}{5}
%---------------------------------------------------------------------------
%---------------------------------------------------------------------------


\begin{document}
%\chapterimage{png/Fond_Cin}
\input{style/new_pagegarde}
\vspace{5cm}
\pagestyle{fancy}
\thispagestyle{plain}


\def\columnseprulecolor{\color{ocre}}
\setlength{\columnseprule}{0.4pt} 

%\ifprof
%\else
\begin{multicols}{2}
%\fi
\section*{Mise en situation}
\textit{La machine de rééducation SYS-REEDUC est issue d'un projet régional entre différents laboratoires de recherche : le CReSTIC (Centre de Recherche en Sciences et Technologies de l'Information et de la Communication) de Reims et le CRITT-MDTS (Centre Régional d'Innovation et de Transfert de Technologie) de Charleville-Mézières. L'objectif de ce projet était de réaliser un système capable d'évaluer et d'aider à la rééducation des membres inférieurs.}


\begin{center}
\includegraphics[width=.8\linewidth]{images/fig_02}
%\textit{}
\end{center}

\begin{obj}
L'objectif de cette partie est de modéliser l'asservissement du système, puis de paramétrer le correcteur pour répondre aux exigences.
\end{obj}

Pour permettre au kinésithérapeute de rééduquer les membres inférieurs du patient, on doit respecter les exigences suivantes : 
\begin{center}
\begin{tabular}{|l|c|}
\hline 
Critère & Niveau \\ \hline\hline
Angle de rotation de la cuisse &  De 0 à 150\degres \\ \hline
Effort du patient & Jusqu'à \SI{20}{N}   \\ \hline
Écart de position & Nul   \\ \hline
Marge de gain & \SI{7}{dB} mini \\ \hline
Marge de phase &  45\degres \\ \hline
Rapidité &  $t_{5\%} < \SI{0,2}{s}$ \\ \hline
Pulsation au gain unité & $\SI{50}{rad.s^{-1}}$\\
\hline
\end{tabular}
\end{center}

La structure du schéma-blocs permettant l'asservissement du déplacement longitudinal du << chariot >> (support mobile) est donnée dans la figure suivante.


\begin{center}
\includegraphics[width=\linewidth]{images/fig_03}
%\textit{}
\end{center}

\subsection*{Éléments de modélisation}
On propose alors une modélisation par schéma-blocs dans la figure suivante. 
\begin{center}
\includegraphics[width=\linewidth]{images/fig_04}
%\textit{}
\end{center}

Le moteur à courant continu est régi par les équations suivantes :
$ u_m(t)=e(t)+Ri(t)$, $e(t)=k_e\omega_m(t)$ et $C_{M1}(t)=k_t i(t)$. 

Une étude dynamique a mené à l'équation suivante : 
$$\left(M+m\right)r\rho_1 \dot{\omega}_m(t)=\dfrac{C_{M1}(t)}{\rho_1 r}-F_p(t)$$ avec : $M$ la masse du chariot et $m$ la masse du support de pied, $\rho_1=\dfrac{1}{10}$ le rapport de réduction du réducteur, $r=\SI{46,1}{mm}$ le rayon de la poulie du transmetteur poulie--courroie, $C_{M1}(t)$ le couple délivré par le moteur et $F_p(t)$ l'effort délivré par le patient sur le support 3. 

Le codeur incrémental possède 500 fentes équiréparties. Deux émetteurs-récepteurs positionnés en quadrature permettent de mesurer l'information. 

\subparagraph{}\textit{À partir des équations proposées, déterminer les fonctions de transfert $K_1$, $K_2$, $H_3(p)$, $H_4(p)$,  $K_5$, $K_6$, $K_7$, $K_8$ et $K_9$.}
\ifprof
\begin{corrige}~\\
\end{corrige}
\else
\fi

\subparagraph{}\textit{Montrer que le schéma-blocs peut être mis sous la forme suivante. On exprimera $A$, $B$ et $D$ en fonction des paramètres du système $r$, $\rho_1$, $k_t$, $k_e$, $R$, $M$, $m$ et $K_8$. }
\ifprof
\begin{corrige}~\\
\end{corrige}
\else
\fi

\begin{center}
\includegraphics[width=\linewidth]{images/fig_05}
%\textit{}
\end{center}

Pour la suite du sujet on gardera les constantes $A$, $B$ et $D$, avec $A=\SI{6700}{m/V}$, $B=\SI{0,01}{s}$ et $D=\SI{6}{N/V}$.


\subsection*{Correction proportionnelle}

On suppose que $C(p)=K_c$. 


\subparagraph{}\textit{Exprimer $\varepsilon_x$ en fonction des deux entrées $F_p$ et $X_c$ et des constantes $A$, $B$, $D$ et $K_c$.}
\ifprof
\begin{corrige}~\\
\end{corrige}
\else
\fi


\subparagraph{}\textit{Tracer le diagramme de Bode de la FTBO du système et donner les marges. Le cahier des charges est-il vérifié ?}
\ifprof
\begin{corrige}~\\
\end{corrige}
\else
\fi

\subparagraph{}\textit{Déterminer l'écart de position $\varepsilon_x$ en réponse à deux échelons d'intensité $F_0$ pour la force du patient et $X_0$ pour le déplacement. Conclure quant au respect du cahier des charges.}
\ifprof
\begin{corrige}~\\
\end{corrige}
\else
\fi

\subsection*{Correction proportionnelle intégrale}
On suppose maintenant que $C(p)=K_i\left(1+\dfrac{1}{T_ip}\right)$

\subparagraph{}\textit{Exprimer $\varepsilon_x$ en fonction des deux entrées$F_p$ et $X_c$ et des constantes $A$, $B$, $D$ et $K_i$.}
\ifprof
\begin{corrige}~\\
\end{corrige}
\else
\fi

\subparagraph{}\textit{Déterminer l'écart de position $\varepsilon_x$ en réponse à deux échelons d'intensité $F_0$ pour la force du patient et $X_0$ pour le déplacement. Conclure quant au respect du cahier des charges.}
\ifprof
\begin{corrige}~\\
\end{corrige}
\else
\fi


\subparagraph{}\textit{Déterminer la fonction de transfert en boucle ouverte du système $FTBO(p)=\dfrac{X}{\varepsilon_x}$ en supposant que $F_p=0$.}
\ifprof
\begin{corrige}~\\
\end{corrige}
\else
\fi

\subparagraph{}\textit{Déterminer la valeur $T_i$ permettant d'assurer la marge de phase pour la pulsation au gain unité souhaitée (pulsation pour laquelle le gain en décibel est nul).}
\ifprof
\begin{corrige}~\\
\end{corrige}
\else
\fi


\subparagraph{}\textit{Déterminer $K_i$ permettant d'assurer la pulsation au gain unité souhaitée.}


\ifprof
\begin{corrige}~\\
\end{corrige}
\else
\fi

On donne sur le document réponse la réponse temporelle du système à une entrée de type échelon unitaire sur le déplacement ($F_p=0$) ainsi que le diagramme de Bode de la FTBO.

\subparagraph{}\textit{Conclure quant au respect du cahier des charges sur le reste des critères énoncés. Faire apparaître sur le document réponse les grandeurs mesurées.}
\ifprof
\begin{corrige}~\\
\end{corrige}
\else
\fi

\end{multicols}

\begin{center}
\includegraphics[width=.5\linewidth]{images/fig_06}
%\textit{}
\end{center}

\begin{center}
\includegraphics[width=.5\linewidth]{images/fig_07}
%\textit{}
\end{center}

\end{document}

\subparagraph{}\textit{}
\ifprof
\begin{corrige}~\\
\end{corrige}
\else
\fi

\begin{center}
\includegraphics[width=\linewidth]{images/img_04}
%\textit{}
\end{center}


\subparagraph{}\textit{}
\ifprof
\begin{corrige}~\\
\end{corrige}
\else
\fi

\subparagraph{}\textit{}
\ifprof
\begin{corrige}~\\
\end{corrige}
\else
\fi

\subparagraph{}\textit{}
\ifprof
\begin{corrige}~\\
\end{corrige}
\else
\fi

\subparagraph{}\textit{}
\ifprof
\begin{corrige}~\\
\end{corrige}
\else
\fi

\subparagraph{}\textit{}
\ifprof
\begin{corrige}~\\
\end{corrige}
\else
\fi

\subparagraph{}\textit{}
\ifprof
\begin{corrige}~\\
\end{corrige}
\else
\fi
\begin{center}
%\includegraphics[width=\linewidth]{images/fig_05}
\end{center}
