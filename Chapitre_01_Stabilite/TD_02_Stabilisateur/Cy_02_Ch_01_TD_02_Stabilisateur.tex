\documentclass[10pt,fleqn]{article} % Default font size and left-justified equations
\usepackage[%
    pdftitle={Modélisation SLCI : Stabilité des systèmes},
    pdfauthor={Xavier Pessoles}]{hyperref}

    
\input{style/new_style}
\input{style/macros_SII}




\fichetrue
%\fichefalse

\fichetrue
%\fichefalse

\proftrue
\proffalse

\tdtrue
%\tdfalse

\courstrue
\coursfalse



\newif\ifnormal
\normaltrue
%\normalfalse

\newif\ifdifficile
\difficilefalse
%\difficiletrue

\newif\iftdifficile
\tdifficilefalse
%\tdifficiletrue


\newif\ifcolle
%\colletrue
\collefalse



\def\discipline{Sciences \\Industrielles de \\ l'Ingénieur}
\def\xxtete{Sciences Industrielles de l'Ingénieur}

\def\classe{\textsf{PSI$\star$ -- MP}}
\def\xxnumpartie{Cycle 02}
\def\xxpartie{Modéliser les systèmes asservis dans le but de prévoir leur comportement}

\def\xxnumchapitre{Chapitre 1 \vspace{.2cm}}
\def\xxchapitre{\hspace{.12cm} Stabilité des systèmes}




\def\xxtitreexo{Stabilisateur actif d'image  \ifnormal $\star$ \else \fi \ifdifficile $\star\star$ \else \fi \iftdifficile $\star\star\star$ \else \fi }
\def\xxsourceexo{\hspace{.2cm} \footnotesize{Mines Ponts 2018 -- PSI}}


\def\xxposongletx{2}
\def\xxposonglettext{1.45}
\def\xxposonglety{20}
%\def\xxonglet{Part. 1 -- Ch. 3}
\def\xxonglet{Cycle 04}

\def\xxactivite{\ifcolle Colle \else TD 1 \fi }
\def\xxauteur{\textsl{Xavier Pessoles}}

\def\xxcompetences{%
\textsl{%
\textbf{Savoirs et compétences :}\\
%\begin{itemize}[label=\ding{112},font=\color{ocre}] 
%\item \textit{Mod2.C13} : centre d'inertie
%\item \textit{Mod2.C14} : opérateur d'inertie
%\item \textit{Mod2.C15} : matrice d'inertie
%\end{itemize}
}}
\def\xxfigures{
\includegraphics[width=.7\linewidth]{images/fig_00}
}%figues de la page de garde


\def\xxpied{%
Cycle 04 -- Modélisation mécanique -- Cinétique\\% afin de valider leurs performances.\\
Chapitre 3 -- \xxactivite%
}

\setcounter{secnumdepth}{5}
%---------------------------------------------------------------------------

\usepackage{pgfplots}
\begin{document}
\def\pathfig{images}
%\chapterimage{png/Fond_Cin}
\input{style/new_pagegarde}
\vspace{5cm}
\pagestyle{fancy}
\thispagestyle{plain}

\def\columnseprulecolor{\color{ocre}}
\setlength{\columnseprule}{0.4pt} 

\def\pathfig{images}

\ifprof
%\begin{multicols}{2}
\else
\begin{multicols}{2}
\fi

\subsection*{Mise en situation}
 \ifprof
 \else



\begin{obj}
Suite à une sollicitation brève de \SI{0,5}{m.s^{-2}}, l'amplitude des oscillations de la caméra ne doit pas dépasser les 0,5\degres.
\end{obj}

\subsection*{Travail demandé}

On considère un modèle de l’axe de tangage sans perturbation et qui reçoit des consignes assez rapides modélisées par
des échelons.
L’ensemble \{moteur, charge\} ne présente pas de réducteur. Il est modélisé par un ensemble en série de deux fonctions
de transfert :
\begin{itemize}
\item un gain pur de valeur $K_m$;% (La valeur KmT du DOCUMENT D4 est notée Km dans cette partie) ;
\item une fonction de transfert du premier ordre de gain statique $A$ et de constante de temps $\tau_m$.
\end{itemize}
Cet ensemble présente comme entrée la commande du moteur $Com(t)$ et comme sortie la vitesse angulaire de rotation
du moteur $\omega_m(t)$. Le réglage retenu est tel que $K_m A = 1$. Le retour $K_D$ agit par un sommateur.

\begin{center}
\includegraphics[width=\linewidth]{images/fig_01}

\textit{Modèle 1 de l’axe de tangage}
\end{center}

\subparagraph{}
\textit{Avec $K_m A = 1$, calculer la fonction de transfert en boucle ouverte (FTBO) et la fonction de transfert
en boucle fermée (FTBF) du schéma (modèle 1).}
\ifprof
\begin{corrige}
\end{corrige}
\else
\fi

Dans un premier temps en mode pilotage, on s’intéresse au comportement de l’axe de tangage sans le filtre passe bas :
$A_1(p)=1$.

\subparagraph{}
\textit{Quelle est la valeur maximale de $K_D$ pour que la commande de l’axe de tangage soit strictement
stable ? Préciser le(s) critère(s) de stabilité appliqué(s).}
\ifprof
\begin{corrige}
\end{corrige}
\else
\fi

En accord avec les résultats précédents, on fixe $K_D = 0,5$ et $\tau_m = \SI{0,2}{s}$.
Dans un premier temps on impose $K_P = \SI{10}{s^{-1}}$.


La figure temporelle ci-dessous propose une réponse du système avec un filtre passe
bas de constante de temps 2 secondes et de gain égal à 1 [i=3].

\begin{center}
\includegraphics[width=\linewidth]{images/fig_02}

\end{center}


\subparagraph{}
\textit{Montrer que le comportement est alors compatible avec l’exigence 1.12 «~Maîtriser les déplacements~» :
«~les mouvements de caméra doivent être réalisés avec départ rapide et arrivée lente sans aucun dépassement~». }
\ifprof
\begin{corrige}
\end{corrige}
\else
\fi


Dans un second temps on se place en mode stabilisation. On s’intéresse toujours au comportement de l’axe de
tangage mais sans le filtre passe bas ($A_1(p)=1$).
On considère ici que la consigne est constante donc $\varphi^*_a(t)=0$. Une perturbation $Pe(p)$ agit au niveau de l’ensemble (moteur, charge) modélisée sur le schéma bloc (Modèle 2). On appelle $Com(p)$ la transformée de Laplace de la commande du moteur $com(t)$.

\begin{center}
\includegraphics[width=\linewidth]{images/fig_03}

\textit{Modèle 2 de l’axe de tangage}
\end{center}


\subparagraph{}
\textit{Avec le « modèle 2 » calculer la fonction de transfert qui lie la commande du $\text{Stab}(p)=\dfrac{\text{Com}(p)}{\text{Pe}(p)}$
qui lie à la perturbation. Conseil de résolution : calculer $\varepsilon_1$ en fonction de $\text{Pe}(p)$, $\text{Com}(p)$ et des fonctions de transfert utiles, puis calculer $\varepsilon_2$ en fonction de $\varepsilon_1$ et des fonctions de transfert utiles, puis $\varepsilon_3$ en fonction de $\varepsilon_1$, $\varepsilon_2$ et des fonctions de transfert utiles et enfin en déduire $\text{Stab}(p)=\dfrac{\text{Com}(p)}{\text{Pe}(p)}$.}
\ifprof
\begin{corrige}
\end{corrige}
\else
\fi

\subparagraph{}
\textit{Avec le modèle 2 et une entrée $\text{Pe}(p)$ échelon unitaire, déterminer la limite quand t tend vers
l’infini de la commande : $\text{com}(t)$. Quel sens physique donner à ce résultat ?}
\ifprof
\begin{corrige}
\end{corrige}
\else
\fi

\subparagraph{}
\textit{Avec le modèle 2 déterminer la FTBO $\dfrac{\text{Mes}\varphi(p)}{\varepsilon_2(p)}$ de ce schéma puis calculer la fonction de transfert liant la perturbation et la sortie $\text{Pert}(p)=\dfrac{\varphi(p)}{\varepsilon_2(p)}$ de ce schéma puis calculer la fonction de transfert liant la perturbation et la sortie $\text{Pert}(p)=\dfrac{\varphi(p)}{\text{Pe}(p)}$.}
\ifprof
\begin{corrige}
\end{corrige}
\else
\fi



\subparagraph{}
\textit{Déterminer la valeur lorsque t tend vers l’infini de la réponse temporelle de ce système à une
perturbation de type échelon unitaire. Quel sens physique donner à ce résultat ?}
\ifprof
\begin{corrige}
\end{corrige}
\else
\fi

\subparagraph{}
\textit{On désire une marge de gain de $M_G \geq \SI{5}{dB}$ et une marge de phase $M\varphi \geq 20\degres$ (OU 40 ?) (exigence 1.14 « Stabilité de la commande »). Déterminer la valeur maximale de $K_P$ en utilisant les données ci-dessous.}
\begin{center}
\includegraphics[width=\linewidth]{images/fig_04}
\end{center}
\ifprof
\begin{corrige}
\end{corrige}
\else
\fi

Le document réponse présente la réponse temporelle de l’axe de tangage à une perturbation sinusoïdale (due par
exemple au vent qui crée un balancement de la GYRCAM).

\subparagraph{}
\textit{Analyser ce tracé par rapport à l’exigence 1.13 « Perturbations » du DOCUMENT D5-a et justifier le
tracé de Com(t) relativement à Pe(t) en utilisant le résultat de la question 28****.}
\ifprof
\begin{corrige}
\end{corrige}
\else
\fi

Afin d’améliorer le comportement, un autre réglage a été effectué, ce qui donne sur le document réponse un nouveau
tracé.

\subparagraph{}
\textit{Analyser comparativement ce nouveau tracé. Quel(s) réglage(s) ont été fait(s) ? Quelle est
l’amélioration principale obtenue ?}
\ifprof
\begin{corrige}
\end{corrige}
\else
\fi



\ifcolle
\else
\vspace{.5cm}
\begin{tabular}{|p{.95\linewidth}|}
\hline
Éléments de corrigé
\begin{enumerate}
\item .
\item .
\end{enumerate}\\
\hline
\end{tabular}
\fi

\ifprof
\else
\end{multicols}
\fi






\end{document}
\begin{center}
\includegraphics[width=\linewidth]{images/}
\end{center}

\subparagraph{}
\textit{}
\ifprof
\begin{corrige}
\end{corrige}
\else
\fi
