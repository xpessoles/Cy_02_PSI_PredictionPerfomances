\documentclass[10pt,fleqn]{article} % Default font size and left-justified equations
\usepackage[%
    pdftitle={Modélisation SLCI : Stabilité des systèmes},
    pdfauthor={Xavier Pessoles}]{hyperref}
    
\input{style/new_style}
\input{style/macros_SII}
\usepackage{multicol}
\usepackage{siunitx}
%\usepackage{picins}
\fichetrue
%\fichefalse

\proftrue
\proffalse

\tdtrue
%\tdfalse

\courstrue
\coursfalse

\def\discipline{Sciences \\Industrielles de \\ l'Ingénieur}
\def\xxtete{Sciences Industrielles de l'Ingénieur}

\def\classe{PSI$\star$ -- MP}
\def\xxnumpartie{Cycle 02}
\def\xxpartie{Modéliser les SLCI dans le but de prévoir leur comportement}


\def\xxnumchapitre{Chapitre 1 \vspace{.2cm}}
\def\xxchapitre{\hspace{.12cm} Stabilité des systèmes}


\def\xxtitreexo{Activation}%Motorisation du moteur Haibike}
\def\xxsourceexo{\hspace{.2cm} \footnotesize{Patrick Dupas, \url{http://patrick.dupas.chez-alice.fr/}.}}


\def\xxposongletx{2}
\def\xxposonglettext{1.45}
\def\xxposonglety{20}
%\def\xxonglet{Part. 1 -- Ch. 3}
\def\xxonglet{Cycle 02}

\def\xxactivite{Activation}
\def\xxauteur{\textsl{P. Dupas.}}

\def\xxcompetences{%
\textsl{%
\textbf{Savoirs et compétences :}\\
%Les sources sont associées par un \emph{hacheur série}. La détermination des grandeurs électriques associées à ce montage permet de conclure vis à vis du cahier des charges.
%\noindent \textbf{Résoudre :} à partir des modèles retenus :
%\begin{itemize}[label=\ding{112},font=\color{ocre}] 
%\item choisir une méthode de résolution analytique, graphique, numérique;
%\item mettre en \oe{}uvre une méthode de résolution.
%\end{itemize}
%\begin{itemize}[label=\ding{112},font=\color{ocre}] 
%\item \textit{Rés -- C1.1 :} Loi entrée sortie géométrique et cinématique -- Fermeture géométrique.
%\end{itemize}
%
%\noindent \textit{Mod2 -- C4.1 :} Représentation par schéma bloc.
}}

\def\xxfigures{
%\includegraphics[width=.9\linewidth]{images/c-evolution}
}%figues de la page de garde


\def\xxpied{%
Cycle 02 -- Modéliser les systèmes afin de prévoir leur comportement\\
Chapitre 1 -- \xxactivite%
}

\setcounter{secnumdepth}{5}
%---------------------------------------------------------------------------

\usepackage{pgfplots}
\begin{document}
\def\pathfig{images}
%\chapterimage{png/Fond_Cin}
\input{style/new_pagegarde}
\vspace{4cm}
\pagestyle{fancy}
\thispagestyle{plain}

\def\columnseprulecolor{\color{ocre}}
\setlength{\columnseprule}{0.4pt} 

\def\pathfig{images}

\ifprof
\else
\begin{multicols}{2}
\fi
\subsection*{Identification de la FTBF}
\begin{obj}
Identifier les caractéristiques de la FTBF d'un système à partir d'une réponse temporelle et fréquentielle. Caractériser la stabilité du système.
\end{obj}


Un système a fait l’objet d’essais temporel et harmoniques.


\subparagraph{}\textit{En utilisant la réponse temporelle, identifier la fonction de transfert du système}
\ifprof
\begin{corrige}
Le premier dépassement a une valeur de 30,4\%. On a
$D_\%=e^{-\dfrac{\pi \xi}{\sqrt{1-\xi^2}}}$
$\Rightarrow \ln D = -\dfrac{\pi \xi}{\sqrt{1-\xi^2}}$
$\Rightarrow \left( \sqrt{1-\xi^2} \right)  = -\dfrac{\pi \xi}{\ln D}$
$\Rightarrow 1-\xi^2= \dfrac{\pi^2 \xi^2}{\left(\ln D\right)^2}$
$\Rightarrow 1= \xi^2\left( 1+\dfrac{\pi^2 }{\left(\ln D\right)^2}\right)$
$\Rightarrow \xi^2= \dfrac{1}{ 1+\dfrac{\pi^2 }{\left(\ln D\right)^2}}$
$\Rightarrow \xi=0,35$.

La pseudo-période est de \SI{0,475}{s}. On a  $T_p=\dfrac{2\pi}{\omega_0\sqrt{1-\xi^2}}$
$\Leftrightarrow \omega_0=\dfrac{2\pi}{T_p\sqrt{1-\xi^2}}$ et $\omega_0=\SI{14,15}{rad.s^{-1}}$.
 
 On a donc $K=1$, $\xi=0,35$ et $\omega_0=\SI{14,15}{rad.s^{-1}}$.
\end{corrige}
\else
\fi

\subparagraph{}\textit{En utilisant la réponse fréquentielle, identifier à nouveau la fonction de transfert du système.}
\ifprof
\begin{corrige}
On observe une réponse harmonique constituée d'une asymptote horizontale quand $\omega$ tend vers 0 et d'une asymptote de pente $-\SI{40}{dB/decade}$. Il en résulte qu'on peut modéliser le système par un second ordre. 

Lorsque $\omega$ tend vers 0, le gain est nul; donc $K=1$.

L'intersection des asymptotes a lieu pour $\omega_0=\SI{14,14}{rad.s^{-1}}$. 

À la résonance, on mesure un gain de $20\log A_{\text{max}}=\SI{3,59}{dB}\Rightarrow A_{\text{max}}=1,51$. On a donc 
$A_{\text{max}} = \dfrac{1}{2\xi\sqrt{1-\xi^2}}$
$\Leftrightarrow {2\xi\sqrt{1-\xi^2}}A_{\text{max}}=1$
$\Rightarrow 4\xi^2\left(1-\xi^2\right)A_{\text{max}}^2=1 $
$\Rightarrow 4\xi^2A_{\text{max}}^2 -4\xi^4 A_{\text{max}}^2-1=0 $
$\Rightarrow \xi^4 - \xi^2 +\dfrac{1}{4A_{\text{max}}^2}=0 $. En posant $\xi^2=X$,  
$ X^2 - X +\dfrac{1}{4A_{\text{max}}^2}=0 $
. On a alors $\Delta = 1-\dfrac{1}{A_{\text{max}}^2}$ et $X_{1,2}=\dfrac{1\pm \sqrt{\Delta}}{2}\simeq \dfrac{1\pm 0,75}{2}$. On a donc $X_1=0,125$, $X_2=0,875$ et $\xi_1=0,35$, $\xi_2=0,94$. Étant donné qu'il existe une résonance, on prend $\xi=0,35$.

 On a donc $K=1$, $\xi=0,35$ et $\omega_0=\SI{14,15}{rad.s^{-1}}$ et $F(p)=\dfrac{1}{1+0,05p+0,005p^2}$.

\end{corrige}
\else
\fi

\begin{rem}
On montrera dans les deux questions précédentes que 
$K=1$, $\xi\simeq0,35$ et $\omega_0\simeq\SI{14,15}{rad.s^{-1}}$.
\end{rem}

\subparagraph{}\textit{Conclure.}
\ifprof
\begin{corrige}
On retrouve les mêmes coefficients. 
\end{corrige}
\else
\fi

\subparagraph{}\textit{Caractériser la stabilité à partir des éléments de la FTBF.}



\ifprof
\begin{corrige}
On a un système du second ordre. Le système est stable. 
\end{corrige}
\else
\fi

\subsection*{Étude de la stabilité}

\begin{obj}
\begin{itemize}
\item Caractériser la stabilité d'un système à partir de la FTBO.
\item La marge de gain est supérieure à $\SI{10}{dB}$ et que la marge de phase est supérieure à \SI{45}{\degres}.
\end{itemize}

\end{obj}

On donne le schéma bloc suivant :

\begin{center}
\includegraphics[width=.6\linewidth]{images/fig_02}
\end{center}

\subparagraph{}\textit{Justifier la forme du schéma-blocs retenu pour modéliser la FTBO qui sera notée $H(p)$.}
\ifprof
\begin{corrige}
On a $H(p)=\dfrac{S(p)}{E(p)}
=\dfrac{\dfrac{KG}{\left(1+\tau p\right)p}}{1+\dfrac{KG}{\left(1+\tau p\right)p}}
=\dfrac{1}{\dfrac{\tau}{KG} p^2+\dfrac{p}{KG}+1}$.

Ainsi, $H(p)$ est un système du second ordre avec un gain unitaire, comme la fonction identifiée dans les premières questions. 
\end{corrige}
\else
\fi

On considère le correcteur proportionnel $K=1$. 
\subparagraph{}\textit{Déterminer les valeurs de $G$ et de $\tau$ et en déduire $H(p)$.}
\ifprof
\begin{corrige}
On a :
$\dfrac{1}{\dfrac{\tau}{G} p^2+\dfrac{p}{G}+1}
= \dfrac{1}{1+0,05p+0,005p^2}$. En conséquences, $G=20$ et $\tau/20=0,005 \Rightarrow \tau=\SI{0,1}{s}$ et $FTBO(p)=\dfrac{20}{(1+0,1p)p}$.

\end{corrige}
\else
\fi

\subparagraph{}\textit{Déterminer l'erreur statique et l'erreur de traînage.}

\subparagraph{}\textit{Effectuer les tracés des diagrammes de Bode de la FTBO.}
\ifprof
\begin{corrige}

\begin{center}
\includegraphics[width=\linewidth]{images/cor_01}
\end{center}
\end{corrige}
\else
\fi
%Le cahier des charges impose une marge de gain de $\SI{10}{dB}$ et une marge de phase de $
%\SI{45}{\degree}$.

\subparagraph{}\textit{Déterminer graphiquement les marges de gains et de phase.}
\ifprof
\begin{corrige}
\end{corrige}
\else
\fi

\subparagraph{}\textit{Confirmer ces résultats par le calcul.}
\ifprof
\begin{corrige}
La phase ne coupe jamais l'axe des abscisses. Ainsi, La marge de gain n'est pas définie (elle est infinie).
Pour déterminer la marge de phase analytiquement :
\begin{enumerate}
\item On cherche $\omega_c$ tel que $G_{\text{dB}}(\omega_c)=0$;
\item On calcule $\varphi(\omega_c)$;
\item La marge de phase est de $\varphi(\omega_c) -(-180)$.
\end{enumerate}

Cherchons $\omega_c$ tel que $G_{\text{dB}}(\omega_c)=0$. 
On a $FTBO(j\omega )
=\dfrac{20}{(1+0,1j\omega)j\omega}
=\dfrac{20}{j\omega-0,1\omega^2}$. 
$20\log |FTBO(j\omega )| 
= 20\log 20 - 20\log \sqrt{\omega^2+0,01\omega^4}
= 20\log 20 - 20\log \omega\sqrt{1+0,01\omega^2}$.

 $G_{\text{dB}}(\omega_c)=0 
\Leftrightarrow   20 =\omega_c\sqrt{1+0,01\omega_c^2} 
\Leftrightarrow   400 =\omega_c^2 \left(1+0,01\omega_c^2\right)$
On pose $x=\omega_c^2$ et on a :
$400 =x \left(1+0,01x\right)\Leftrightarrow x^2+100x-40000=0$. 
On a donc $\Delta = 412,3^2$ et $x_{1,2}=\dfrac{-100\pm412,3}{2}$ on conserve la racine positive et  $x_1=156,15$ et $\omega_c=\SI{12,5}{rad.s^{-1}}$.

$\varphi(\omega_c)=\arg(20)-90 -\arg\left(1+0,1j\omega_c \right)
=0- 90 -\arctan\left( 0,1\omega_c \right)
=0-90-51,34=-141,34\degres$.

La marge de phase est donc de 38,66\degres.
\end{corrige}
\else
\fi

\subparagraph{}\textit{Conclure par rapport au cahier des charges.}
\ifprof
\begin{corrige}
Le système ne sera pas stable vis-à-vis du cahier des charges.
\end{corrige}
\else
\fi

\subsection*{Choix d'un gain}

\begin{obj}
Déterminer le gain permettant de satisfaire le cahier des charges.
\end{obj}

\subparagraph{}\textit{Déterminer graphiquement la valeur du correcteur $K$ à placer ans la chaîne directe, afin de respecter les critères de stabilité du cahier des charges.}

\subparagraph{}\textit{Déterminer analytiquement la valeur du correcteur $K$ à placer ans la chaîne directe, afin de respecter les critères de stabilité du cahier des charges.}
\ifprof
\begin{corrige}
On a une phase de -135\degres pour $\omega=\SI{10}{rad.s^{-1}}$. Il faut donc déterminer $K$ tel que le gain soit nul en $\omega=\SI{10}{rad.s^{-1}}$.

$20\log 20 - 20\log 10\sqrt{1+0,01\cdot 10^2}=20\log 20 - 20\log 10\sqrt{2}=\SI{3}{dB}$. Il faut donc diminuer le gain de \SI{3}{dB}.

On cherche donc $K$ tel que $20 \log K = -3$ et il faut prendre $K=0,7$.
\end{corrige}
\else
\fi

\subparagraph{}\textit{Quel sera alors le 1\ier dépassement pour la réponse indicielle du système ?}
\ifprof
\begin{corrige}
Dépassement de 23\%.
\end{corrige}
\else
\fi

\ifprof
\else
\end{multicols}
\fi

\noindent\begin{minipage}[c]{.4\linewidth}
\begin{center}
\includegraphics[width=\linewidth]{images/fig_03}
\end{center}
\end{minipage}\hfill
\begin{minipage}[c]{.46\linewidth}
\begin{center}
\includegraphics[width=.9\linewidth]{images/fig_01}
\end{center}
\end{minipage}

\begin{center}
\includegraphics[width=.7\linewidth]{images/img_04}
\end{center}

%
%\begin{center}
%\includegraphics[width=\linewidth]{images/fig_02}
%\end{center}
%
%\begin{center}
%\includegraphics[width=\linewidth]{images/fig_03}
%\end{center}
%
%\begin{center}
%\includegraphics[width=.8\linewidth]{images/fig_04}
%\end{center}



\end{document}