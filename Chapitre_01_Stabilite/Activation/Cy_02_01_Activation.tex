\documentclass[10pt,fleqn]{article} % Default font size and left-justified equations
\usepackage[%
    pdftitle={Modélisation SLCI : Stabilité des systèmes},
    pdfauthor={Xavier Pessoles}]{hyperref}
    
\input{style/new_style}
\input{style/macros_SII}
\usepackage{multicol}
\usepackage{siunitx}
%\usepackage{picins}
\fichetrue
%\fichefalse

\proftrue
%\proffalse

\tdtrue
%\tdfalse

\courstrue
\coursfalse

\def\discipline{Sciences \\Industrielles de \\ l'Ingénieur}
\def\xxtete{Sciences Industrielles de l'Ingénieur}

\def\classe{PSI$\star$ -- MP}
\def\xxnumpartie{Cycle 02}
\def\xxpartie{Modéliser les systèmes asservis dans le but de prévoir leur comportement}


\def\xxnumchapitre{Chapitre 1 \vspace{.2cm}}
\def\xxchapitre{\hspace{.12cm} Stabilité des systèmes}


\def\xxtitreexo{Activation}%Motorisation du moteur Haibike}
\def\xxsourceexo{\hspace{.2cm} \footnotesize{Patrick Dupas, \url{http://patrick.dupas.chez-alice.fr/}.}}


\def\xxposongletx{2}
\def\xxposonglettext{1.45}
\def\xxposonglety{20}
%\def\xxonglet{Part. 1 -- Ch. 3}
\def\xxonglet{Cycle 02}

\def\xxactivite{Activation}
\def\xxauteur{\textsl{P. Dupas.}}

\def\xxcompetences{%
\textsl{%
\textbf{Savoirs et compétences :}\\
%Les sources sont associées par un \emph{hacheur série}. La détermination des grandeurs électriques associées à ce montage permet de conclure vis à vis du cahier des charges.
%\noindent \textbf{Résoudre :} à partir des modèles retenus :
%\begin{itemize}[label=\ding{112},font=\color{ocre}] 
%\item choisir une méthode de résolution analytique, graphique, numérique;
%\item mettre en \oe{}uvre une méthode de résolution.
%\end{itemize}
%\begin{itemize}[label=\ding{112},font=\color{ocre}] 
%\item \textit{Rés -- C1.1 :} Loi entrée sortie géométrique et cinématique -- Fermeture géométrique.
%\end{itemize}
%
%\noindent \textit{Mod2 -- C4.1 :} Représentation par schéma bloc.
}}

\def\xxfigures{
%\includegraphics[width=.9\linewidth]{images/c-evolution}
}%figues de la page de garde


\def\xxpied{%
Cycle 02 -- Modéliser les SLCI dans le but de prévoir leur comportement\\
Chapitre 1 -- \xxactivite%
}

\setcounter{secnumdepth}{5}
%---------------------------------------------------------------------------

\usepackage{pgfplots}
\begin{document}
\def\pathfig{images}
%\chapterimage{png/Fond_Cin}
\input{style/new_pagegarde}
\vspace{6cm}
\pagestyle{fancy}
\thispagestyle{plain}

\def\columnseprulecolor{\color{ocre}}
\setlength{\columnseprule}{0.4pt} 

\def\pathfig{images}

\begin{multicols}{2}
\subsection*{Identificaiton de la FTBF et de la FTBO -- Etude de la stabilité}

Un système a fait l’objet d’essais en temporel et en harmonique.
\begin{center}
\includegraphics[width=\linewidth]{images/fig_01}
\end{center}
\begin{center}
\includegraphics[width=\linewidth]{images/fig_03}
\end{center}



\subparagraph*{}\textit{En utilisant la réponse temporelle, identifier la fonction de transfert du système}

\subparagraph*{}\textit{En utilisant la réponse fréquentielle, identifier à nouveau la fonction de transfert du système.}


\subparagraph*{}\textit{Conclure.}

\subparagraph*{}\textit{Caractérisier la stabilité à partir des éléments de la FTBF.}

On donne le schéma bloc suivant :

\begin{center}
\includegraphics[width=\linewidth]{images/fig_02}
\end{center}

\subparagraph*{}\textit{Justifier la forme du schéma-blocs retenu pour modéliser la FTBO qui sera notée $H(p)$.}

On sonsidère le correcteur poroportionnel $K=1$. 
\subparagraph*{}\textit{Déterminer les vaelurs de $G$ et de $\tau$ et en déduire $H(p)$.}
\end{multicols}

\begin{center}
\includegraphics[width=\linewidth]{images/fig_02}
\end{center}

\begin{center}
\includegraphics[width=\linewidth]{images/fig_03}
\end{center}

\begin{center}
\includegraphics[width=.8\linewidth]{images/fig_04}
\end{center}



\end{document}