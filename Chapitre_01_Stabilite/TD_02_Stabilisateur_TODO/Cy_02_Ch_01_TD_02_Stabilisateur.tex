\documentclass[10pt,fleqn]{article} % Default font size and left-justified equations
\usepackage[%
    pdftitle={Modélisation SLCI : Stabilité des systèmes},
    pdfauthor={Xavier Pessoles}]{hyperref}

    
\input{style/new_style}
\input{style/macros_SII}




\fichetrue
%\fichefalse

\fichetrue
%\fichefalse

\proftrue
\proffalse

\tdtrue
%\tdfalse

\courstrue
\coursfalse


% TD PLutot difficile de base

\newif\ifnormal
%\normaltrue
\normalfalse

\newif\ifdifficile
\difficilefalse
\difficiletrue

\newif\iftdifficile
\tdifficilefalse
%\tdifficiletrue


\newif\ifcolle
%\colletrue
\collefalse



\def\discipline{Sciences \\Industrielles de \\ l'Ingénieur}
\def\xxtete{Sciences Industrielles de l'Ingénieur}

\def\classe{\textsf{PSI$\star$ -- MP}}
\def\xxnumpartie{Cycle 02}
\def\xxpartie{Modéliser les systèmes asservis dans le but de prévoir leur comportement}

\def\xxnumchapitre{Chapitre 1 \vspace{.2cm}}
\def\xxchapitre{\hspace{.12cm} Stabilité des systèmes}




\def\xxtitreexo{Stabilisateur actif d'image  \ifnormal $\star$ \else \fi \ifdifficile $\star\star$ \else \fi \iftdifficile $\star\star\star$ \else \fi }
\def\xxsourceexo{\hspace{.2cm} \footnotesize{Mines Ponts 2018 -- PSI}}


\def\xxposongletx{2}
\def\xxposonglettext{1.45}
\def\xxposonglety{20}
%\def\xxonglet{Part. 1 -- Ch. 3}
\def\xxonglet{Cycle 04}

\def\xxactivite{\ifcolle Colle \else TD 1 \fi }
\def\xxauteur{\textsl{Xavier Pessoles}}

\def\xxcompetences{%
\textsl{%
\textbf{Savoirs et compétences :}\\
%\begin{itemize}[label=\ding{112},font=\color{ocre}] 
%\item \textit{Mod2.C13} : centre d'inertie
%\item \textit{Mod2.C14} : opérateur d'inertie
%\item \textit{Mod2.C15} : matrice d'inertie
%\end{itemize}
}}
\def\xxfigures{
\includegraphics[width=.7\linewidth]{images/fig_00}
}%figues de la page de garde


\def\xxpied{%
Cycle 04 -- Modélisation mécanique -- Cinétique\\% afin de valider leurs performances.\\
Chapitre 3 -- \xxactivite%
}

\setcounter{secnumdepth}{5}
%---------------------------------------------------------------------------

\usepackage{pgfplots}
\begin{document}
\def\pathfig{images}
%\chapterimage{png/Fond_Cin}
\input{style/new_pagegarde}
\vspace{5cm}
\pagestyle{fancy}
\thispagestyle{plain}

\def\columnseprulecolor{\color{ocre}}
\setlength{\columnseprule}{0.4pt} 

\def\pathfig{images}

\ifprof
%\begin{multicols}{2}
\else
\begin{multicols}{2}
\fi

\subsection*{Mise en situation}
On s'intéresse à une nacelle active de caméra. Ce système de stabilisation, nommé CAM-GYR, permet de s'assurer que quelque soit l'orientation du porteur (caméraman), l'axe vertical de la caméra et toujours parallèle à la direction de la pesanteur. 
Le système est équipé de 3 moteurs permettant d'ajuster le roulis, le tangage et le lacet. On s'intéresse ici uniquement à la stabilisation de l'axe de tangage. 

\begin{center}
\includegraphics[width=\linewidth]{images/Exigences}
\end{center}

\begin{obj}
Justifier la forme particulière de la commande asservie (ou régulée) de la caméra stabilisée et vérifier l'exigence 1.1 « déplacer la caméra ». %Proposer un réglage d’un axe suivant plusieurs modes de fonctionnement.
\end{obj}


\subsection*{Travail demandé}

On considère un modèle de l’axe de tangage sans perturbation et qui reçoit des consignes assez rapides modélisées par
des échelons.
L’ensemble \{moteur, charge\} ne présente pas de réducteur. Il est modélisé par un ensemble en série de deux fonctions
de transfert :
\begin{itemize}
\item un gain pur de valeur $K_m$;% (La valeur KmT du DOCUMENT D4 est notée Km dans cette partie) ;
\item une fonction de transfert du premier ordre de gain statique $A$ et de constante de temps $\tau_m$.
\end{itemize}
Cet ensemble présente comme entrée la commande du moteur $\text{com(t)}$ et comme sortie la vitesse angulaire de rotation
du moteur $\omega_m(t)$. Le réglage retenu est tel que $K_m A = 1$. \textbf{Le retour $K_D$ agit par un sommateur.}
Dans cette étude, $A_i(p)=1$.
%Enfin, lors de mouvement brusque de la caméra, on souhaite arriver progressivement sur la scène finale sans choc ni oscillation mais avec précision. Ainsi, la commande peut être modifiée selon 3 cas de fonctionnements : 
%\begin{itemize}
%\item $A_1(p)=1$ : pas de traitement;
%\item $A_2(p)= \dfrac{1}{\tau_0 p} \left( 1-e^{-\tau_0 p}\right)$ : fonction rampe de pente 1 pour rejoindre la valeur finale;
%\item $A_3(p)= \dfrac{1}{1+\tau_0 p}$ : fonction de transfert du  premier ordre de constante de temps $\tau$ (filtre passe-bas).
%\end{itemize}
\begin{center}
\includegraphics[width=\linewidth]{images/fig_01}

\textit{Modèle 1 de l’axe de tangage}
\end{center}

\subparagraph{}
\textit{Avec $K_m A = 1$, calculer la fonction de transfert en boucle ouverte (FTBO) et la fonction de transfert
en boucle fermée (FTBF) du schéma (modèle 1).}
\ifprof
\begin{corrige}
\textbf{Attention au signe du comparateur de la boucle inbriquée !}

On définit la FTBO par $\text{FTBO}(p)=\dfrac{\varepsilon(p)}{\text{Mes}\varphi(p)}$ avec $\varepsilon(p)$ la sortie du premier comparateur.

On a d'une part $G(p)=\dfrac{\dfrac{K_m A}{1+\tau_m p}}{1-\dfrac{K_m A K_D }{1+\tau_m p}}= \dfrac{K_m A}{1+\tau_m p-K_m A K_D }$.
On a alors $\text{FTBO}(p)=\dfrac{K_m A K_P}{p\left(1+\tau_m p-K_m A K_D \right)}$. 

Si on définit la FTBF par $\text{FTBF(p)}=\dfrac{\varphi(p)}{\varphi^{\star}(p)}$, on a 
$\text{FTBF(p)}=A_i(p)\dfrac{\dfrac{K_m A K_P}{p\left(1+\tau_m p-K_m A K_D \right)}}{1+\dfrac{K_m A K_P}{p\left(1+\tau_m p-K_m A K_D \right)}}$ 

$=A_i(p)\dfrac{K_m A K_P}{p\left(1+\tau_m p-K_m A K_D \right)+K_m A K_P}$.

Au final, 
$\text{FTBO}(p)=\dfrac{ K_P}{p\left(1+\tau_m p-K_D \right)}$ et 
$\text{FTBF}(p)=A_i(p)\dfrac{K_P}{p\left(1+\tau_m p- K_D \right)+K_P}$.
\end{corrige}
\else
\fi

Dans un premier temps en mode pilotage, on s’intéresse au comportement de l’axe de tangage sans le filtre passe bas :
$A_1(p)=1$.

\subparagraph{}
\textit{Quelle est la valeur maximale de $K_D$ pour que la commande de l’axe de tangage soit strictement
stable ? Préciser le(s) critère(s) de stabilité appliqué(s).}
\ifprof
\begin{corrige}
Pour que le système soit stable, tous les coefficients du dénominateur $D(p)$ de la FTBF doivent être de même signe (ainsi toutes les racines sont à partie réelle négative). On a 
$D(p)=p\left(1+\tau_m p- K_D \right)+K_P = \tau_m p^2 p+ \left(1-K_D \right)p +K_P$ et donc nécessairement, 
$1-K_D >0$ et $K_D < 1$.
\end{corrige}
\else
\fi

En accord avec les résultats précédents, on fixe $K_D = 0,5$ et $\tau_m = \SI{0,2}{s}$.
Dans un premier temps on impose $K_P = \SI{10}{s^{-1}}$.

%
%La figure temporelle ci-dessous propose une réponse du système avec un filtre passe
%bas de constante de temps 2 secondes et de gain égal à 1 ($A_i(p)=\dfrac{1}{1+\tau_0 p}$).%'[i=3].
%
%\begin{center}
%\includegraphics[width=\linewidth]{images/fig_02}
%\end{center}


\subparagraph{}
\textit{Lorsque $A_i(p)=1$, le comportement est-il compatible avec l’exigence 1.12 «~Maîtriser les déplacements~» :
«~les mouvements de caméra doivent être réalisés avec départ rapide et arrivée lente sans aucun dépassement~». }
\ifprof
\begin{corrige}
On a : $\text{FTBF}(p)=
\dfrac{K_P}{p+\tau_m p^2- K_Dp +K_P}$ 
$=\dfrac{K_P}{\dfrac{\tau_m}{K_p} p^2 + p\dfrac{1- K_D}{K_P} +1}$.

On a alors $\omega_0 = \sqrt{\dfrac{K_P}{\tau_m}}$ et $\xi = \dfrac{1- K_D}{K_P} \dfrac{\sqrt{\dfrac{K_P}{\tau_m}}}{2}= \dfrac{1- K_D}{2\sqrt{K_P \tau_m}}= \dfrac{0,5}{2\sqrt{2}} <1$. Il y a donc du dépassement. L'exigence n'est pas vérifiée. 
\end{corrige}
\else
\fi


Dans un second temps on se place en mode stabilisation. On s’intéresse toujours au comportement de l’axe de
tangage mais sans le filtre passe bas ($A_1(p)=1$).
On considère ici que la consigne est constante donc $\varphi^*_a(t)=0$. Une perturbation $\text{Pe(p)}$ agit au niveau de l’ensemble (moteur, charge) modélisée sur le schéma bloc (Modèle 2). On appelle $\text{Com(p)}$ la transformée de Laplace de la commande du moteur $\text{com(t)}$.

\begin{center}
\includegraphics[width=\linewidth]{images/fig_03}

\textit{Modèle 2 de l’axe de tangage}
\end{center}


\subparagraph{}
\textit{Avec le « modèle 2 » calculer la fonction de transfert qui lie la commande du $\text{Stab}(p)=\dfrac{\text{Com}(p)}{\text{Pe}(p)}$
qui lie à la perturbation.}% Conseil de résolution : calculer $\varepsilon_1$ en fonction de $\text{Pe}(p)$, $\text{Com}(p)$ et des fonctions de transfert utiles, puis calculer $\varepsilon_2$ en fonction de $\varepsilon_1$ et des fonctions de transfert utiles, puis $\varepsilon_3$ en fonction de $\varepsilon_1$, $\varepsilon_2$ et des fonctions de transfert utiles et enfin en déduire $\text{Stab}(p)=\dfrac{\text{Com}(p)}{\text{Pe}(p)}$.}
\ifprof
\begin{corrige}
On a $\varepsilon_2(p) = -\text{Mes}\left( \varphi(p)\right) = -\varphi(p) = -\varepsilon_1(p)\dfrac{1}{p}$. 
Par ailleurs, $\varepsilon_1(p)=\text{Pe}(p)+\varepsilon_3(p)\dfrac{AK_m}{1+\tau_m p}$. 
Enfin, $\varepsilon_3(p)=K_P\varepsilon_2(p)+K_D \varepsilon_1(p)$ $\Leftrightarrow \varepsilon_3(p)=\varepsilon_1(p)\left(K_D-\dfrac{K_P}{p} \right)$
$\Leftrightarrow \varepsilon_1(p) =\varepsilon_3(p)\dfrac{1}{K_D -\dfrac{K_P}{p} }$. 

On a donc  $\varepsilon_3(p)\dfrac{1}{K_D -\dfrac{K_P}{p}}=\text{Pe}(p)+\varepsilon_3(p)\dfrac{AK_m}{1+\tau_m p} $ $\Leftrightarrow \varepsilon_3(p)\left(\dfrac{p}{pK_D-K_P } -\dfrac{AK_m}{1+\tau_m p}\right)=\text{Pe}(p) $ 

$\Leftrightarrow \varepsilon_3(p)\dfrac{p\left(1+\tau_m p \right) - AK_m \left( pK_D-K_P\right) }{\left(pK_D-K_P \right) \left(1+\tau_m p\right)}=\text{Pe}(p) $.

On a donc $\text{Stab}(p)=\dfrac{\text{Com}(p)}{\text{Pe}(p)}=\dfrac{\left(pK_D-K_P \right) \left(1+\tau_m p\right)}{p\left(1+\tau_m p \right) - AK_m \left( pK_D-K_P\right) }$.



 
\end{corrige}
\else
\fi

\subparagraph{\label{q28}}
\textit{Avec le modèle 2 et une entrée $\text{Pe}(p)$ échelon unitaire, déterminer la limite quand t tend vers
l’infini de la commande : $\text{com}(t)$. Quel sens physique donner à ce résultat ?}
\ifprof
\begin{corrige}
On a $\lim\limits_{t\to\infty}\text{com}(t) = \lim\limits_{p\to 0}p\text{Com}(p)$ 
$=\lim\limits_{p\to 0}p\text{Stab}(p)\text{Pe}(p)$ 

  $= \lim\limits_{p\to 0}p\dfrac{1}{p} \dfrac{\left(pK_D-K_P \right) \left(1+\tau_m p\right)}{p\left(1+\tau_m p \right) - AK_m \left( pK_D-K_P\right) }$
   $= \lim\limits_{p\to 0} \dfrac{-K_P }{  AK_m  K_P }=-1$ si $AK_m=1$.
  
  Ainsi, pour une perturbation angulaire dans un autre sens, le système commande les moteurs avec une consigne dans le sens opposé. 
\end{corrige}
\else
\fi

\subparagraph{}
\textit{Avec le modèle 2 déterminer la FTBO $\dfrac{\text{Mes}\varphi(p)}{\varepsilon_2(p)}$ de ce schéma puis calculer la fonction de transfert liant la perturbation et la sortie $\text{Pert}(p)=\dfrac{\varphi(p)}{\text{Pe}(p)}$.}
\ifprof
\begin{corrige}

On a  $\dfrac{\text{Mes}\varphi(p)}{\varepsilon_2(p)} = \dfrac{K_m A K_P}{p\left(1+\tau_m p-K_m A K_D \right)}$ (c'est la même que pour le premier modèle).

On a vu que  $\varepsilon_2(p) = -\varphi(p) = -\varepsilon_1(p)\dfrac{1}{p}$, $\varepsilon_1(p)=\text{Pe}(p)+\varepsilon_3(p)\dfrac{AK_m}{1+\tau_m p}$ et $\varepsilon_3(p)=\varepsilon_1(p)\left(K_D-\dfrac{K_P}{p} \right)$. 

En conséquences, $\varepsilon_1(p)=\text{Pe}(p)+\varepsilon_3(p)\dfrac{AK_m}{1+\tau_m p} \Longleftrightarrow 
\varepsilon_1(p)=\text{Pe}(p)+\varepsilon_1(p)\left(K_D-\dfrac{K_P}{p} \right)\dfrac{AK_m}{1+\tau_m p}$

$ \Leftrightarrow 
\varepsilon_1(p)\left( 1+\left(\dfrac{K_P}{p}-K_D \right)\dfrac{AK_m}{1+\tau_m p}\right)=\text{Pe}(p)$
$ \Leftrightarrow 
p\varphi(p)\left( 1+\left(\dfrac{K_P}{p}-K_D \right)\dfrac{AK_m}{1+\tau_m p}\right)=\text{Pe}(p)$

et donc $\text{Pert}(p)=\dfrac{1}{p\left( 1+\left(\dfrac{K_P}{p}-K_D \right)\dfrac{AK_m}{1+\tau_m p}\right)}$
$=\dfrac{1}{p\left( 1+\dfrac{K_P-pK_D}{p}\dfrac{AK_m}{1+\tau_m p}\right)}$
$=\dfrac{1+\tau_m p}{p \left(1+\tau_m p\right)+\left(K_P-pK_D\right)AK_m}$.


\end{corrige}
\else
\fi



\subparagraph{}
\textit{Déterminer la valeur lorsque t tend vers l’infini de la réponse temporelle de ce système à une
perturbation de type échelon unitaire. Quel sens physique donner à ce résultat ?}
\ifprof
\begin{corrige}
On a $\lim\limits_{t\to\infty}\varphi(t) = \lim\limits_{p\to 0}p\Phi(p)$ 
$=\lim\limits_{p\to 0}p\text{Pert}(p)\text{Pe}(p)$
  $= \lim\limits_{p\to 0}p\dfrac{1}{p} \dfrac{1+\tau_m p}{p \left(1+\tau_m p\right)+\left(K_P-pK_D\right)AK_m}$
  
  $= \lim\limits_{p\to 0}\dfrac{1}{K_PAK_m} =\dfrac{1}{K_P}=0,1\degres$.
  
Le système n'est pas précis s'il y a une perturbation échelon. 
\end{corrige}
\else
\fi

\subparagraph{}
\textit{On désire une marge de gain de $M_G \geq \SI{5}{dB}$ et une marge de phase $M\varphi \geq 20\degres$ (exigence 1.14 « Stabilité de la commande »). Déterminer la valeur maximale de $K_P$ en utilisant les données ci-dessous.}
% (OU 40 ?degres)


\footnotesize
On note $F(\omega)= \dfrac{2}{j\omega \left(1+0,4 j\omega \right)}$.
\begin{center}
\begin{tabular}{|c|c|c|c|c|c|}
\hline
$\omega$ (rad/s) & 1 & 2,5 & 5 & 7 & 10 \\ \hline
$\text{Arg}\left(F(\omega)\right)$  & $-112\degres$ & $-135\degres$ & $-153\degres$ & $-160\degres$ & $-166\degres$ \\ \hline
$20 \log \left| F(\omega)\right|$ & \SI{5,4}{dB} & \SI{3}{dB} & \SI{-1}{dB} & \SI{-3}{dB} & \SI{-6,2}{dB} \\ \hline
\end{tabular}
\end{center}
\normalsize

%\begin{center}
%\includegraphics[width=\linewidth]{images/fig_04}
%\end{center}
\ifprof
\begin{corrige}
Pour une marge  de de phase de 20\degres, la phase doit être de $-160\degres$ lorsque le gain est nul. Or en $-160\degres$ le gain est de \SI{-3}{dB}. Pour respecter la marge de phase, il faut donc déterminer $K_P$ tel que $20\log K_P = 3$ soit $K_P < 10^{\dfrac{3}{20}}\simeq1,41 $.

Le système étant d'ordre 2, la marge de gain sera forcément infinie.
\end{corrige}
\else
\fi

Le figure suivante présente la réponse temporelle de l’axe de tangage à une perturbation sinusoïdale (due par
exemple au vent qui crée un balancement de la GYRCAM) (ordonnée en degrés).


\begin{center}
\includegraphics[width=\linewidth]{images/fig_05}
\end{center}

\subparagraph{}
\textit{Analyser ce tracé par rapport à l’exigence 1.13 « Perturbations » du DOCUMENT D5-a et justifier le
tracé de Com(t) relativement à Pe(t) en utilisant le résultat de la question \ref{q28}.}
\ifprof
\begin{corrige}
La commande s'oppose à la perturbation (comme évoqué question \ref{q28}). Le stabilisateur a au final un mouvement sinusoïdal dont les valeurs maximales et minimales sont voisines de 0,1\degres et $-0,1\degres$. 
\end{corrige}
\else
\fi

Afin d’améliorer le comportement, un autre réglage a été effectué (voir figure suivante).

\begin{center}
\includegraphics[width=\linewidth]{images/fig_06}
\end{center}


\subparagraph{}
\textit{Analyser comparativement ce nouveau tracé.}% Quel(s) réglage(s) ont été fait(s) ? Quelle est l’amélioration principale obtenue ?}
\ifprof
\begin{corrige}
Dans ce cas, les mouvements du porteur sont inférieurs à 0,1 degres (en valeur absolue).
\end{corrige}
\else
\fi



\ifcolle
\else
\vspace{.5cm}
\begin{tabular}{|p{.95\linewidth}|}
\hline
Éléments de corrigé
\begin{enumerate}
\item .
\item .
\end{enumerate}\\
\hline
\end{tabular}
\fi

\ifprof
\else
\end{multicols}
\fi






\end{document}
\begin{center}
\includegraphics[width=\linewidth]{images/}
\end{center}

\subparagraph{}
\textit{}
\ifprof
\begin{corrige}
\end{corrige}
\else
\fi
