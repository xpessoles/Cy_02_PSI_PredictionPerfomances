\documentclass[10pt,fleqn]{article} % Default font size and left-justified equations
\usepackage[%
    pdftitle={Modélisation SLCI : Rapidité des systèmes},
    pdfauthor={Xavier Pessoles}]{hyperref}

\input{style/new_style}
\input{style/macros_SII}

\fichetrue
%\fichefalse

\proftrue
%\proffalse

%\tdtrue
\tdfalse

%\courstrue
\coursfalse

% -------------------------------------
% Déclaration des titres
% -------------------------------------

\def\discipline{Sciences \\Industrielles de \\ l'Ingénieur}
\def\xxtete{Sciences Industrielles de l'Ingénieur}

\def\classe{\textsf{Cy 02}}
\def\xxnumpartie{Cycle 02}
\def\xxpartie{Modéliser les systèmes asservis dans le but de prévoir leur comportement}

\def\xxnumchapitre{Chapitre 1 \vspace{.2cm}}
\def\xxchapitre{\hspace{.12cm} Rapidité des systèmes}

\def\xxposongletx{2}
\def\xxposonglettext{1.45}
\def\xxposonglety{19}%16

\def\xxonglet{Cycle 02}

\def\xxactivite{Fiche}
\def\xxauteur{\textsl{Xavier Pessoles}}

\def\xxcompetences{%
\textsl{%
\textbf{Savoirs et compétences :}\\
}}

\def\xxfigures{
%incgraphics[width=.8\textwidth]{}%images/prot_01}
}%figues de la page de garde

\def\xxpied{%
Cycle 02 -- Modéliser les SLCI dans le but de prévoir leur comportement\\
Chapitre 2 -- \xxactivite%
}

\setcounter{secnumdepth}{5}
%---------------------------------------------------------------------------


\begin{document}
%\chapterimage{png/Fond_Cin}
\input{style/new_pagegarde}
\vspace{2cm}
\pagestyle{fancy}
\thispagestyle{plain}

\section{Amortissement}
La notion d'amortissement est associé \textbf{à la boucle fermée}. Cette notion caractérise :  
\item la façon dont les oscillations décroissent pour un système oscillatoire;
\item le temps de stabilisation pour un système non oscillatoire.
\end{itemize}
\begin{resultat}
En observant la réponse d'un système dans le domaine temporel, plus le dépassement $D_1$ est important, moins le système est amorti.
\end{resultat}

\begin{resultat}
En observant la réponse fréquentielle du système, il en résulte que plus la surtension est importante, moins le système est amorti.
\end{resultat}

\begin{resultat}
Dans le lieu des pôles, plus les pôles sont dans une position correspondant à un angle $\varphi$ grand, plus le système est amorti.
\end{resultat}

\section{Rapidité}
\end{document}


